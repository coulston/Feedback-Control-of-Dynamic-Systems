\documentclass[11pt]{amsart}
\usepackage[width=6.5in, height=9in]{geometry}                % See geometry.pdf to learn the layout options. There are lots.
\geometry{letterpaper}                   % ... or a4paper or a5paper or ... 
%\geometry{landscape}                % Activate for for rotated page geometry
%\usepackage[parfill]{parskip}    % Activate to begin paragraphs with an empty line rather than an indent
\usepackage{tikz}
\usetikzlibrary{arrows}
\usetikzlibrary{circuits.ee.IEC}
\usetikzlibrary{shapes.geometric}
\usetikzlibrary{decorations.pathreplacing}
\usetikzlibrary{patterns}
\usepackage{framed}
\usetikzlibrary{decorations.pathmorphing}

\usepackage{graphics}
\usepackage{amsmath}
\usepackage{amsfonts}
\DeclareGraphicsRule{.tif}{png}{.png}{`convert #1 `dirname #1`/`basename #1 .tif`.png}

\begin{document}

\begin{minipage}{3in}\noindent\texttt{s=tf([1 0],1); % define Laplace Variable
figure(1)\\
bode(1/(s\^{}2+s+1))\\
}\end{minipage}\begin{minipage}{3in}
\includegraphics[height=2.5in]{bodeprob_labeled}\\
\end{minipage}\\
From the Bode Plot, $|G(j2)| = 10^{-11/20} = .28$ and $\angle G(j2) = -146$. Thus at steady state
\[
\theta(t) = .28\cos(2t -146)
\]
This is close to the answer we got using the linear approximation, but not exact. Note that the actual phase slope is higher than $-90 \circ$/dec when the damping ratio is small ($\zeta=0.5$ in this case)
\end{document}  

