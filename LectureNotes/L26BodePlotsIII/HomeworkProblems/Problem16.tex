
Consider the following system, with transfer function given below. \\
\begin{minipage}{4in}\begin{center}
\resizebox{4in}{!}{\begin{tikzpicture}[scale=1.3,inner sep=0pt,outer sep=0pt,very thick]
\draw (0,0) node[fill=black] (a) {}; 
\draw (3.5,0) node[fill=black] (c) {};
\draw (0,-2) node[fill=black] (d) {};
\draw (2,-2) node[fill=black] (e) {};
\draw (3.5,-2) node[fill=black] (f) {};
 
\draw (1,0) node (R1) {\input{\mainfolder/DrawingElements/CircuitElements/resistor.tex}};
\draw (1,0) node[above=.2in] {$R_{a}$};
\draw (2.5,0) node (L1) {\input{\mainfolder/DrawingElements/CircuitElements/inductor.tex}};
\draw (2.5,0) node[above=.2in] {$L_{a}$};
\draw (3.5,-1) node (Rot) {\input{\mainfolder/DrawingElements/CircuitElements/rotor.tex}}; 
\draw (5,-1) node (I) {\input{\mainfolder/DrawingElements/MechanicalElements/inertia2.tex}};
\draw (I) node {$J$};
\draw[->] (4,-1) ++(0,.5)  node[above=2pt] {$\theta$}  .. controls  ++(-.15,-.3) and ++(-.15,.3) ..  ++(0,-1);
\draw[->] (4.25,-1) ++(0,.5) node[above=2pt] {$\tau_{m}$}  .. controls  ++(-.15,-.3) and ++(-.15,.3) ..  ++(0,-1);
\draw (6.5,-1) node (D) {\begin{tikzpicture}
\draw[very thick] (-.2,0) -- (0,0);
\draw (.75,0) node {\begin{tikzpicture}
\draw[very thick] (-.2,0) -- (0,0);
\draw (.75,0) node {\begin{tikzpicture}
\draw[very thick] (-.2,0) -- (0,0);
\draw (.75,0) node {\input{\mainfolder/DrawingElements/MechanicalElements/damper.tex}};
\draw (.75,0) node[above=9pt] {$b$};
\draw[very thick] (1.5,0) -- ++(.2,0);
    \draw[<-,thick] (1.5,0) ++(.2,0) -- ++(.5,0) node[right] {$f$};
    \draw[<-,thick] (-.2,0) -- ++(-.5,0) node[left] {$f$};
    \draw[|->,thick] (-.2,.4) node[above=2pt] {$x_{1}$} -- ++(.5,0);  
    \draw[|->,thick] (1.7,.4) node[above=2pt] {$x_{2}$} -- ++(.5,0);  
    \draw (.6,-.6) node {$x=x_{1}-x_{2}$};
  %  \draw (.6,-1.2) node {$f=b\dot{x}$};
\end{tikzpicture}};
\draw (.75,0) node[above=9pt] {$b$};
\draw[very thick] (1.5,0) -- ++(.2,0);
    \draw[<-,thick] (1.5,0) ++(.2,0) -- ++(.5,0) node[right] {$f$};
    \draw[<-,thick] (-.2,0) -- ++(-.5,0) node[left] {$f$};
    \draw[|->,thick] (-.2,.4) node[above=2pt] {$x_{1}$} -- ++(.5,0);  
    \draw[|->,thick] (1.7,.4) node[above=2pt] {$x_{2}$} -- ++(.5,0);  
    \draw (.6,-.6) node {$x=x_{1}-x_{2}$};
  %  \draw (.6,-1.2) node {$f=b\dot{x}$};
\end{tikzpicture}};
\draw (.75,0) node[above=9pt] {$b$};
\draw[very thick] (1.5,0) -- ++(.2,0);
    \draw[<-,thick] (1.5,0) ++(.2,0) -- ++(.5,0) node[right] {$f$};
    \draw[<-,thick] (-.2,0) -- ++(-.5,0) node[left] {$f$};
    \draw[|->,thick] (-.2,.4) node[above=2pt] {$x_{1}$} -- ++(.5,0);  
    \draw[|->,thick] (1.7,.4) node[above=2pt] {$x_{2}$} -- ++(.5,0);  
    \draw (.6,-.6) node {$x=x_{1}-x_{2}$};
  %  \draw (.6,-1.2) node {$f=b\dot{x}$};
\end{tikzpicture}};
\draw (D) node[above=12pt] {$b$};
\draw (7.5,-1) node[rotate=180] (gnd) {\input{\mainfolder/DrawingElements/MechanicalElements/ground.tex}};

\draw (3.5,-1) node[left=.3in] {$\begin{matrix} + \\ v_{b} \\ -\end{matrix}$};
\draw (0,-1) node (V) {\input{\mainfolder/DrawingElements/CircuitElements/voltagesource.tex}};
\draw (0,-1) node[left=.3in]{$v_{a}$};


%\draw[->] (1.5,-.5) -- node[pos=.5,below=4pt] {$i$} ++(1,0); 
\draw (I.0) -- (D.180);
\draw (Rot.0) -- (I.180);
\draw (D.0) -- (gnd);
\draw (a) -- (R1);
\draw (R1) -- (L1);
\draw (L1) -| (Rot);
\draw (Rot) -- (f);
\draw (f) -- (d);
\draw (a) -- (V);
\draw (d) -- (V);
\end{tikzpicture}}
\end{center}
\end{minipage}\hspace{.25in}
\begin{minipage}{2in}
\[
\frac{\theta(s)}{V_{a}(s)} = \frac{10}{s(s+ 10)(s+0.2)}
\]
\end{minipage}
\begin{enumerate}[(a)] 
\item Using the linear approximation rules, Sketch the bode plot. For full credit, the magnitude and phase at each break frequency must be indicated, and the frequency scale correctly indicated.
\begin{flushright}
\includegraphics[width=5in]{\mainfolder/LectureNotes/\lecturefolder/HomeworkProblems/Problem16/blankbode}
\end{flushright}
\item Using the information from your linear approximation, estimate the steady state output if the input to the system is $r(t)=\cos(10t)$.
\end{enumerate}
