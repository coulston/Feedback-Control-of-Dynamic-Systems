\mode<presentation>
{
  \usetheme{CambridgeUS}
  \usecolortheme{whale}
  \usecolortheme{lily}

  \setbeamercovered{transparent}
  \usefonttheme[onlymath]{serif}
}


\title[Complex Numbers] % (optional, use only with long paper titles)
{Complex Number Review}

\subtitle
{} % (optional)

\author% (optional, use only with lots of authors)
{Tyrone Vincent}
%{T. Vincent\inst{1} \and S.~Another\inst{2}}
% - Use the \inst{?} command only if the authors have different
%   affiliation.

\institute[CSM] % (optional, but mostly needed)
{Division of Engineering\\
  Colorado School of Mines}
% - Use the \inst command only if there are several affiliations.
% - Keep it simple, no one is interested in your street address.

\date % (optional)
{}

% If you have a file called "university-logo-filename.xxx", where xxx
% is a graphic format that can be processed by latex or pdflatex,
% resp., then you can add a logo as follows:

%\pgfdeclareimage[height=1.1cm]{university-logo}{UniversityLogo}
%\logo{\pgfuseimage{university-logo}}



% Delete this, if you do not want the table of contents to pop up at
% the beginning of each subsection:
%\AtBeginSection[]
%{
%  \begin{frame}<beamer>{Outline}
%    \tableofcontents[currentsection,currentsubsection]
%  \end{frame}
%}


% If you wish to uncover everything in a step-wise fashion, uncomment
% the following command:

%\beamerdefaultoverlayspecification{<+->}


\begin{document}

\begin{frame}
  \titlepage
\end{frame}

\mode<article>{
\maketitle
\tableofcontents
}


\section{Complex Number Definition and Representation}

\begin{frame}{Complex \#'s}
\tryit{}{\begin{itemize}
\item Complex numbers are needed to express all roots of polynomials.
\item What is the solution to
\[
x^2+1=0 \text{?}
\]
\item By definition
\[
x=j:=\sqrt{-1}
\]
\end{itemize}
\begin{block}{Complex Numbers}
\[
s=a+jb
\]
where $a$ and $b$ are real numbers
\end{block}}
\end{frame}


\begin{frame}{Complex \#'s and the Complex Plane}
\begin{columns}[c]
\begin{column}{.5\textwidth}
\tryit{\begin{tikzpicture}[>=stealth]
\draw[->] (-3,0) -- (3,0) node[below] {Re$\{s\}$};
\draw[->] (0,-3) -- (0,3) node[left] {Im$\{s\}$};
\end{tikzpicture}}{\input{figures/complexplane1.movie}}
\end{column}
\begin{column}{.4\textwidth}
\tryit
{\begin{block}{Two Representations}
\begin{itemize}%[<+->]
\item Rectangular: \hspace{1in}
\item Polar: 
\end{itemize}
\end{block}
\vspace{1.5in}}
{\begin{block}{Two Representations}
\begin{itemize}%[<+->]
\item Rectangular: $s=a+jb$
\item Polar: $s=r\angle\theta$
\end{itemize}
\end{block}
\begin{block}{Notation}
\begin{itemize}
\item Magnitude: $r=|s|$
\item Angle: $\theta=\angle s$
\end{itemize}
\end{block}
}
\vspace{1cm}
\mbox{ }
\end{column}
\end{columns}
\end{frame}

\begin{frame}{Converting between Rectangular and Polar Form}{Polar to Rectangular}
\hspace{.5in}1$^{\mbox{st}}$ and 4$^{\mbox{th}}$ quadrants \hspace{.75in} 2$^{\mbox{nd}}$ and 3$^{\mbox{rd}}$ quadrants 
\begin{center}
\resizebox{1.75in}{!}{\input{figures/complexplane1.movie}}\hspace{.6in}
\resizebox{1.75in}{!}{\begin{tikzpicture}
\draw[->] (-2.5,0) -- (2.5,0) node[below] {Re$\{s\}$};
\draw[->] (0,-2.5) -- (0,2.5) node[left] {Im$\{s\}$};
\draw[color=blue] (0,0) -- node[pos=.5,color=black,above right] {$r$} (-1.5,1.75) node[circle,draw=blue,fill=blue,inner sep=2.5pt] (s) {};
\draw (s.180) node[left] {$s$};
\draw (-1.5,-.1) node[below] {$a$} -- (-1.5,.1);
\draw (-.1,1.75) node[left] {$jb$} -- (.1,1.75);
\draw (.4,0) arc (0:130.6:.4);
\draw (.2,.2)  node[above right] {$\theta$} ;

\end{tikzpicture}}
\end{center}
\tryit{\vspace{.75in}}{
\begin{align*}
a=&r\cos(\theta), \\
b=&r\sin(\theta),
\end{align*}
}
\end{frame}

\begin{frame}{Converting between Rectangular and Polar Form}{Rectangular to Polar}
\hspace{.5in}1$^{\mbox{st}}$ and 4$^{\mbox{th}}$ quadrants \hspace{.75in} 2$^{\mbox{nd}}$ and 3$^{\mbox{rd}}$ quadrants 
\begin{center}
\resizebox{1.75in}{!}{\input{figures/complexplane1.movie}}\hspace{.6in}
\resizebox{1.75in}{!}{\begin{tikzpicture}
\draw[->] (-2.5,0) -- (2.5,0) node[below] {Re$\{s\}$};
\draw[->] (0,-2.5) -- (0,2.5) node[left] {Im$\{s\}$};
\draw[color=blue] (0,0) -- node[pos=.5,color=black,above right] {$r$} (-1.5,1.75) node[circle,draw=blue,fill=blue,inner sep=2.5pt] (s) {};
\draw (s.180) node[left] {$s$};
\draw (-1.5,-.1) node[below] {$a$} -- (-1.5,.1);
\draw (-.1,1.75) node[left] {$jb$} -- (.1,1.75);
\draw (.4,0) arc (0:130.6:.4);
\draw (.2,.2)  node[above right] {$\theta$} ;

\end{tikzpicture}}
\end{center}
\tryit{\vspace{.75in}}{
\begin{minipage}{1.75in}\begin{align*}
r=&\sqrt{a^2+b^2} \\
\theta=&\tan^{-1}\left(\frac{b}{a}\right) 
\end{align*}
\end{minipage}\hspace{.6in}
\begin{minipage}{1.75in}\begin{align*}
r=&\sqrt{a^2+b^2}\\
\theta  = &\pi + \tan^{-1}(b/a)
\end{align*}\end{minipage}
}
\end{frame}


\section{Complex Algebra I - Rectangular Form}


\begin{frame}{Complex Algebra I - Rectangular Form}
\begin{itemize}
\item  $s_1=a_1+jb_1$, $s_2=a_2+jb_2$
\begin{eqnarray*}
s_1+s_2&=&(a_1+a_2)+j(b_1+b_2) \\
s_1-s_2&=&(a_1-a_2)+j(b_1-b_2) \\
s_1s_2&=&(a_1a_2-b_1b_2)+j(a_1b_2+b_1a_2) \\
\frac{s_1}{s_2}&=&\frac{(a_1a_2+b_1b_2)}{(a_2^2+b_2^2)}+j\frac{(b_1a_2-a_1b_2)}{(a_2^2+b_2^2)}
\end{eqnarray*}
\item Complex Conjugate: $s=a+jb$
\[
s^{*}=a-jb
\]
\end{itemize}
\end{frame}



\begin{frame}{Euler's Formula}
\tryit{}{
\[
\boxed{e^{j\theta}=\cos(\theta)+j\sin(\theta)}
\]
\begin{center}
\includegraphics[height=6cm]{figures/complexplaneeuler.pdf}
\end{center}}
\end{frame}


\begin{frame}{Euler's Formula and Polar Form}
\tryit{}{
\[
s = r \angle \theta = r e^{j\theta}
\]
\begin{center}
\includegraphics[height=6cm]{figures/complexplaneeuler.pdf}
\end{center}}
\end{frame}



\begin{frame}{Complex Algebra II - Polar Form}{Multiplication and Division}
\begin{itemize}
\item $s_1=r_1e^{j\theta_1}$, $s_2=r_2e^{j\theta_2}$
\tryit{\vspace{4.5cm}}{\begin{align*}
s_1s_2=&r_1e^{j\theta_1}r_2e^{j\theta_2} \\
=&r_1r_2e^{j\theta_1}e^{j\theta_2} \\
=&r_1r_2e^{j(\theta_1+\theta_2)} \\
\frac{s_1}{s_2}=&\frac{r_1e^{j\theta_1}}{r_2e^{j\theta_2}}\\
=&\frac{r_1}{r_2}e^{j(\theta_1-\theta_2)}
\end{align*}}
\item Conjugate: Polar Form: $s=re^{j\theta}$
\tryit{\vspace{1.5cm}}{\[
s^*=re^{-j\theta}
\]}
\end{itemize}
\end{frame}

\begin{frame}{Example}
Find the following in polar form:\vspace{.1in}\\
 $(1+j)2e^{j\pi/2}=$\vspace{.2in}
\tryit{\vspace{3in}}{First, note that
\[
(1+j) = \sqrt{2}e^{j\pi/4}
\]
Thus
\[
(1+j)2e^{j\pi/2} = \sqrt{2}e^{j\pi/4}2e^{j\pi/2} = 2\sqrt{2} e^{j3\pi/4}
\]
}
\end{frame}

\begin{frame}{Example}
Find the magnitude of the following complex number:\vspace{.1in}\\
$\frac{(1+j)(2+2j)}{3+4j}$\vspace{.2in}
\tryit{\vspace{3in}}{\input{examplesolutions/polar2}}
\end{frame}

\begin{frame}{Example}
Find the phase of the following complex number\vspace{.1in}\\
$\frac{(1+j)(2+2j)}{3+4j}$\vspace{.2in}
\tryit{\vspace{3in}}{The fact that when complex numbers multiply, their magnitudes add (and subtract for division) means that we can solve this problem by finding the phases of each term in the expression {\em first} and then adding (if multiplied) or subtracting (if divided).
\begin{align*}
\angle \frac{(1+j)(2+2j)}{3+4j} &= \angle (1+j) + \angle (2+2j) - \angle (3+4j)\\
& = \frac{\pi}{4} + \frac{\pi}{4} - 0.927 = .644 \mbox{ (rad)}\\
 &= 45^{\circ} + 45^{\circ} - 53 =  37^{\circ}\\
\end{align*}
}
\end{frame}



\end{document}


