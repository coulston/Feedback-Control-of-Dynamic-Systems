The fact that when complex numbers multiply, their magnitudes add (and subtract for division) means that we can solve this problem by finding the phases of each term in the expression {\em first} and then adding (if multiplied) or subtracting (if divided).
\begin{align*}
\angle \frac{(1+j)(2+2j)}{3+4j} &= \angle (1+j) + \angle (2+2j) - \angle (3+4j)\\
& = \frac{\pi}{4} + \frac{\pi}{4} - 0.927 = .644 \mbox{ (rad)}\\
 &= 45^{\circ} + 45^{\circ} - 53 =  37^{\circ}\\
\end{align*}
