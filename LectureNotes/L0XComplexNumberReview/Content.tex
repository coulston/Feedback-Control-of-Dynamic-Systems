\mode<presentation>
{
  \usetheme{CambridgeUS}
  \usecolortheme{whale}
  \usecolortheme{lily}

  \setbeamercovered{transparent}
  \usefonttheme[onlymath]{serif}
}
\externaldocument{main}

\title[Complex Numbers] % (optional, use only with long paper titles)
{Complex Number Review\license}

\subtitle
{} % (optional)

\author[\instructorshort]% (optional, use only with lots of authors)
{\instructorlong\credits}
%{T. Vincent\inst{1} \and S.~Another\inst{2}}
% - Use the \inst{?} command only if the authors have different
%   affiliation.

\institute[\instituteshort] % (optional, but mostly needed)
{\institutelong}

\date{}

% Delete this, if you do not want the table of contents to pop up at
% the beginning of each subsection:
\AtBeginSection[]
{
  \begin{frame}<beamer>{Outline}
    \tableofcontents[currentsection,currentsubsection]
  \end{frame}
}


% If you wish to uncover everything in a step-wise fashion, uncomment
% the following command:

%\beamerdefaultoverlayspecification{<+->}


\begin{document}

\begin{frame}
  \titlepage
\end{frame}

\mode<article>{
\maketitle
\tableofcontents
}

\subsection*{Elements to Review}
These notes assume that you are familiar with the following facts:
\begin{itemize}
\item Quadratic Formula: How do we find the roots of the polynomial
\[
p(x)=ax^2+bx+c?
\]
\item Trig Functions: How can we use $\tan^{-1}(\theta)$ to find the angles of a right triangle?
\item Algebra of exponentials: $e^{a}e^{b}=e^{a+b}$, $e^{a}/e^{b}=e^{a-b}$, $(e^{a})^{n}=e^{an}$
\item
Taylor Series Expansion (MATH112). What are the Taylor Series expansions of $\sin(x)$, $\cos(x)$ and $e^{x}$ at $x=0$?
\end{itemize}

\subsection*{Lesson Objectives}
\begin{itemize}
\item Convert from rectangular to polar form and vice versa
\item Perform the operations of complex conjugate, addition, subtraction, multiplication, division.
\item Use Euler's formula to represent a complex number in polar form
\item Find roots and powers of a complex number using polar form
\end{itemize}

\mode<presentation>{
\begin{frame}{Outline}
  \tableofcontents
  % You might wish to add the option [pausesections]
\end{frame}}


\section{Complex Number Definition and Representation}

\begin{frame}{Definition of Complex Numbers}
\begin{itemize}
\item Complex numbers are needed to express all roots of polynomials.
\item What is the solution to
\[
x^2=-1 \text{?}
\]
\item By definition
\[
j:=\sqrt{-1}
\]
\end{itemize}
\begin{block}<2->{Complex Numbers}
\[
s=a+jb
\]
where $a$ and $b$ are real numbers
\end{block}
\end{frame}

%%\begin{frame}{Fundamental Theorem of Algebra}
%%\begin{block}{Theorem}
%%An $n$th order polynominal has $n$ roots.
%%\end{block}
%%\begin{block}{Example}<2->
%%\[
%%x^3+x^2+x+1=0
%%\]
%%
%%Claim: roots are $x=-1,\, j,\, -j$
%%
%%\mode<presentation>{
%%\begin{itemize}
%%\item $x=-1$ : \only<3>{$(-1)^3+(-1)^2+(-1)+1$}\only<4->{$(-1)^3+(-1)^2+(-1)+1=-1+1-1+1=0$}
%%\item $x=j$\mbox{  } : \only<5>{$(j)^3+(j)^2+(j)+1=j(j)^2+j^2+j+1=0$}\only<6->{$(j)^3+(j)^2+(j)+1=j(-1)-1+j+1=0$}
%%\item $x=-j$ : \only<7>{$(-j)^3+(-j)^2+(-j)+1=-j(-j)^2+(-j)^2-j+1=0$}\only<8->{$(-j)^3+(-j)^2+(-j)+1=-j(-1)-1-j+1=0$}
%%\end{itemize}}
%%\mode<article>{
%%\begin{itemize}
%%\item $x=-1$ : $(-1)^3+(-1)^2+(-1)+1=-1+1-1+1=0$
%%\item $x=j$ : $(j)^3+(j)^2+(j)+1=j(-1)-1+j+1=0$
%%\item $x=-j$ : $(-j)^3+(-j)^2+(-j)+1=-j(-1)-1-j+1=0$
%%\end{itemize}}
%%\
%%\end{block}
%%\end{frame}

\begin{frame}{Complex Plane}
\begin{center}
Visualize complex numbers in the complex plane
\end{center}
\begin{columns}[c]
\begin{column}{.5\textwidth}
\mode<presentation>{
\includegraphics[height=6cm]{figures/complexplane.pdf}}
\mode<article>{
\begin{center}
\includegraphics[height=6cm]{figures/complexplane.pdf}
\end{center}}
\end{column}
\begin{column}{.4\textwidth}
\begin{block}{Two Representations}
\begin{itemize}%[<+->]
\item Rectangular: $s=a+jb$
\item Polar: $s=r\angle\theta$
\end{itemize}
\end{block}
\begin{block}{Notation}
\begin{itemize}
\item Magnitude: $r=|s|$
\item Angle: $\theta=\angle s$
\end{itemize}
\end{block}
\mode<presentation>{

\vspace{1cm}

\mbox{ }}
\end{column}
\end{columns}
\end{frame}

\mode<article>{\subsection{Converting between Rectangular and Polar Form}}
\begin{figure}[ht]
\begin{center}
\includegraphics[height=6cm]{figures/complexplane.pdf}\hspace{1cm}
\includegraphics[height=6cm]{figures/complexplanesecondquadrant.pdf}

\mbox{(a)\hspace{7.0cm}(b)}
\end{center}
\caption{The complex plane. $s$ marks a particular location, which can be identified using either rectangular or polar coordinates. (a) $s$ in first quadrant. (b) $s$ in second quadrant.\label{fig:rectopolar}}
\end{figure}
The two representations of a complex number are
\begin{itemize}
\item Rectangular: $s=a+jb$
\item Polar: $s=r\angle\theta$
\end{itemize}
Where $r=|s|$ is called the magnitude of the complex number, and $\theta=\angle s$ is called the angle or phase of the complex number. In Figure \ref{fig:rectopolar} a complex number is plotted in the complex plane with both representations noted.
Our first objective is to use trigonometry to find the relationships between $a,b,r,\theta$.

\begin{itemize}
\item \textbf{From Polar to Rectangular}: From Figure \ref{fig:rectopolar} we see that
\begin{eqnarray*}
a&=&r\cos(\theta), \\
b&=&r\sin(\theta),
\end{eqnarray*}
giving the transformation from polar to rectangular coordinates. This relationship is valid in all four quadrants.

\item \textbf{From Rectangular to Polar}:
For points in the first quadrant, we see that
\begin{eqnarray*}
\tan(\theta)&=&\left(\frac{b}{a}\right) \\
r^2&=&a^2+b^2
\end{eqnarray*}
Thus the transformation from rectangular to polar becomes
\begin{eqnarray*}
\theta&=&\tan^{-1}\left(\frac{b}{a}\right) \\
r&=&\sqrt{a^2+b^2}
\end{eqnarray*}
The equation for $r$ is valid for all points, but the equation for $\theta$ is not valid 2nd and 3rd quadrants. Note that in these quadrants
\begin{eqnarray*}
\tan(\pm\pi-\theta)&=&\left(\frac{b}{-a}\right) \\
r^2&=&a^2+b^2
\end{eqnarray*}
Thus, in general, the four quadrant inverse tangent rule must be applied. The general rectangular to polar transformation is\\
\[
\boxed{\begin{aligned}
r&=\sqrt{a^2+b^2}, \\
\theta&=
\begin{cases}
\tan^{-1}(b/a), & \mbox{if }a\geq 0 \\
\pm\pi + \tan^{-1}(b/a), &  \mbox{if } a < 0.
\end{cases}
\end{aligned}}
\]
Note that we have used the fact that $\tan^{-1}(-\theta)=-\tan^{-1}(\theta)$.
\end{itemize}

Here are some examples to try:
\begin{example}\label{ex:rectopolar1}
Convert from polar to rectangular
\begin{enumerate}
\item $1 \angle \pi/3$\vspace{.2in}
\item $2 \angle \pi$\vspace{.2in}
\end{enumerate}
\tryit{}{
Solution:
\begin{enumerate}
\item $1 \angle \pi/3$
\[
1 \angle \pi/3 =\cos(\pi/3)+j\sin(\pi/3)=\frac{1}{2}+\frac{\sqrt{3}}{2}j
\]
\item $2 \angle \pi$
\[
2 \angle \pi=2\cos(\pi)+j2\sin(\pi)=-2
\]
\end{enumerate}

}
\end{example}


\begin{example}\label{ex:rectopolar2}
Convert from rectangular to polar
\begin{enumerate}
\item $1+j$\vspace{.2in}
\item $-1+j$\vspace{.1in}
\end{enumerate}
\tryit{}{Solution:
\begin{enumerate}
\item $1+j$
\begin{eqnarray*}
r&=&\sqrt{1+1}=\sqrt{2}\\
\theta&=&\tan^{-1}(1)=\pi/4
\end{eqnarray*}
\item $-1+j$
\begin{eqnarray*}
r&=&\sqrt{1+1}=\sqrt{2}\\
\theta&=&\pi-\tan^{-1}(1)=3\pi/4
\end{eqnarray*}
\end{enumerate}
}
\end{example}

\section{Complex Algebra I - Rectangular Form}

\begin{itemize}
\item Complex Conjugate: $s=a+jb$
\[
s^{*}=a-jb
\]
The complex conjugate of $s$ has the same real part, but negative imaginary part.

\item Addition, Subtraction, Multiplication and Division: Rectangular Form: $s_1=a_1+jb_1$, $s_2=a_2+jb_2$
\begin{eqnarray*}
s_1+s_2&=&(a_1+a_2)+j(b_1+b_2) \\
s_1-s_2&=&(a_1-a_2)+j(b_1-b_2) \\
s_1s_2&=&(a_1+jb_1)(a_2+jb_2) \\
&=&(a_1a_2-b_1b_2)+j(a_1b_2+b_1a_2) \\
\frac{s_1}{s_2}&=&\frac{(a_1+jb_1)}{(a_2+jb_2)} \\
&=&\frac{(a_1+jb_1)(a_2-jb_2)}{(a_2+jb_2)(a_2-jb_2)} \\
&=&\frac{(a_1a_2+b_1b_2)}{(a_2^2+b_2^2)}+j\frac{(b_1a_2-a_1b_2)}{(a_2^2+b_2^2)}
\end{eqnarray*}

Addition and subtraction are easy in rectangular form, but multiplication and division are easier in polar form, which we discuss later.
\end{itemize}

\begin{example}\label{ex:complexalgebra1}
Evaluate the following if $s_{1}=1+j$, $s_{2}=e^{j\pi/3}$
\begin{enumerate}
\item $s_{1}+s_{1}^{*}$\vspace{.2in}
\item $s_{1}-s_{2}$\vspace{.1in}
\end{enumerate}
\tryit{}{Solution:
\begin{enumerate}
\item 
\[
s_{1}+s_{1}^{*} = (1+j) + (1-j) = 2
\]
\item Since we are subtracting, we need to convert both numbers to rectangular form. Note that 
\[
s_{2} = e^{j\pi/3} = \cos(\pi/3) + j \sin(\pi/3) = \frac{1}{2} + j\frac{\sqrt{3}}{2}
\]
Thus,
\[
s_{1} - s_{2} = (1+j) - \left(\frac{1}{2} + j\frac{\sqrt{3}}{2}\right) = \frac{1}{2} + j\frac{2-\sqrt{3}}{2}
\]
\end{enumerate}
}
\end{example}


\section{Euler's Formula and Polar Form}
\begin{itemize}
\item A very important equation that will help make complex multiplication and division easier is Euler's formula. It helps define the extension of the exponential function to complex numbers.
\[
\boxed{e^{j\theta}=\cos(\theta)+j\sin(\theta)}
\]
\end{itemize}
The function $e^{x}$ is very important, as it appears in many places in engineering subjects. The reason that it is so ubiquitous is that is satisfies the relationship
\[
\frac{d}{dx}e^{x}=e^{x}
\]
and thus appears in situations when quantities increase in proportion to how much is already there.
We want to extend this function to complex numbers. In this case, we will need to calculate $e^{z}$ when $z$ is complex. We can build $e^{z}$ through it's Taylor series expansion around $z=0$, since we know $e^{0}=1$, and we want $\frac{d}{dz}e^{z}=e^{z}$. (Item to be swept under the rug: what exactly is the complex derivative of a complex function? Answer: same definition as for real derivatives, but with some extra care: See advanced engineering math!) Thus,
\begin{eqnarray*}
e^{z}&=&e^{0}+\left.\frac{d}{dz}e^{z}\right|_{x=0}z + \frac{1}{2!}\left.\frac{d^2}{dz^2}e^{z}\right|_{z=0}z^2+ \frac{1}{3!}\left.\frac{d^3}{dz^3}e^{z}\right|_{z=0}z^3+ \cdots \\
&=&e^{0}+e^{0}z+\frac{1}{2!}e^{0}z^2+\frac{1}{3!}e^{0}z^3+\cdots \\
&=&1+z+\frac{1}{2!}z^2+\frac{1}{3!}z^3+\cdots
\end{eqnarray*}
We won't verify this, but you can show with a bit of algebra that this function satisfies the laws of exponents
\[
e^{z_1+z_2}=e^{z_1}e^{z_2}.
\]
Thus, if we plug in $z=a+jb$,
\[
e^{a+jb}=e^{a}e^{jb}.
\]
We already know $e^{a}$, so let's find out what $e^{jb}$ is. Using the series expansion
\begin{eqnarray*}
e^{jb}&=&1+jb+\frac{1}{2!}(jb)^2+\frac{1}{3!}(jb)^3+\frac{1}{4!}(jb)^4\cdots \\
&=&1+jb-\frac{1}{2!}(b)^2-j\frac{1}{3!}(b)^3+\frac{1}{4!}(b)^4\cdots \\
&=&(1-\frac{1}{2!}(b)^2+\frac{1}{4!}(b)^4-\cdots)+j(b-\frac{1}{3!}(b)^3+\cdots)
\end{eqnarray*}
But wait, we recognize the following Taylor Series Expansions:
\begin{eqnarray*}
\cos(b)&=&1-\frac{1}{2!}(b)^2+\frac{1}{4!}(b)^4-\cdots \\
\sin(b)&=&b-\frac{1}{3!}(b)^3+\cdots
\end{eqnarray*}
Plugging in gives us \textbf{Euler's Formula}:
\[
e^{jb}=\cos(b)+j\sin(b)
\]
If we plot $e^{jb}$, we see that it lies on the unit circle in the complex plane, at an angle of $b$.

\begin{frame}{Plotting Euler's formula}
\begin{center}
\includegraphics[height=6cm]{figures/complexplaneeuler.pdf}
\end{center}
\end{frame}

Thus, $e^{jb}=1\angle b$. This gives us an alternate way of writing polar form:
\[
\boxed{r\angle\theta = re^{j\theta}}
\]

\begin{example}\label{ex:euler1}
Use Euler's formula to verify that $\frac{d}{d\theta}e^{j\theta}=je^{j\theta}$\\
\rule{0pt}{12pt}\vspace{.3in}
\tryit{}{\input{examplesolutions/euler1}}
\end{example}

\begin{example}\label{ex:euler2}
Use Euler's formula to find $j^{j}$\vspace{.1in}\\
\rule{0pt}{12pt}\vspace{.3in}
\tryit{}{Solution: 
\[
j^{j} = (e^{j\pi/2})^{j} = (e^{j(j\pi/2)}) = e^{-\pi/2} 
\]}
\end{example}


\section{Complex Algebra II - Polar Form}
\mode<article>{\subsection{Multiplication and Division}}
Addition and Subtraction must be done in rectangular form. But the other operations are easier when working in polar form.

\begin{itemize}
\item Conjugate: Polar Form: $s=re^{j\theta}$
\[
s^*=re^{-j\theta}
\]


\item Multiplication and Division: Polar Form: $s_1=r_1e^{j\theta_1}$, $s_2=r_2e^{j\theta_2}$
\begin{eqnarray*}
s_1s_2&=&r_1e^{j\theta_1}r_2e^{j\theta_2} \\
&=&r_1r_2e^{j\theta_1}e^{j\theta_2} \\
&=&r_1r_2e^{j(\theta_1+\theta_2)} \\
\frac{s_1}{s_2}&=&\frac{r_1e^{j\theta_1}}{r_2e^{j\theta_2}}\\
&=&\frac{r_1}{r_2}e^{j(\theta_1-\theta_2)}
\end{eqnarray*}
\end{itemize}

\begin{example}\label{ex:polar1}
Find the following in polar form:
 $(1+j)2e^{j\pi/2}$\vspace{.2in}\\
\tryit{}{First, note that
\[
(1+j) = \sqrt{2}e^{j\pi/4}
\]
Thus
\[
(1+j)2e^{j\pi/2} = \sqrt{2}e^{j\pi/4}2e^{j\pi/2} = 2\sqrt{2} e^{j3\pi/4}
\]

}
\end{example}

\begin{example}\label{ex:polar2}
Find the magnitude of the following complex number: $\frac{(1+j)(2+2j)}{3+4j}$\vspace{.2in}\\
\tryit{}{\input{examplesolutions/polar2}
}
\end{example}

\begin{example}\label{ex:polar3}
Find the phase of the following complex number:
$\frac{(1+j)(2+2j)}{3+4j}$\vspace{.2in}\\
\tryit{}{The fact that when complex numbers multiply, their magnitudes add (and subtract for division) means that we can solve this problem by finding the phases of each term in the expression {\em first} and then adding (if multiplied) or subtracting (if divided).
\begin{align*}
\angle \frac{(1+j)(2+2j)}{3+4j} &= \angle (1+j) + \angle (2+2j) - \angle (3+4j)\\
& = \frac{\pi}{4} + \frac{\pi}{4} - 0.927 = .644 \mbox{ (rad)}\\
 &= 45^{\circ} + 45^{\circ} - 53 =  37^{\circ}\\
\end{align*}

}


\end{example}


\mode<article>{\subsection{Powers and Roots of Complex Numbers}}
Rounding out our complex algebra:
\begin{itemize}
\item Powers: Polar Form: $s=re^{j\theta}$
\[
s^{n}=r^{n}e^{jn\theta}
\]

\item Roots: Polar Form: solve
\[
z^{n}=c.
\]
Let's substitute $z=re^{j\theta}$ for $z$. Then
\[
r^{n}e^{jn\theta}=c,
\]
Since the magnitude and angle of the left and right hand sides must be equal, we get the two equations
\begin{eqnarray*}
r &=& |c|^{1/n}\\
n\theta &=& \angle c + 2\pi\ell \,\,\,\, \mbox{for any integer }\ell.
\end{eqnarray*}
There are $n$ unique solutions to the angle equation, so that
\[
\boxed{z=|c|^{1/n}e^{j\left(\frac{\angle c}{n}+\frac{2\pi\ell}{n}\right)} \,\,\, \ell=0,1,\dots,n}
\]
\end{itemize}

%\ifthenelse{\boolean{PrintExample}}{
\begin{example}
\item Solve the following:
\[
z^{3}=-1
\]
Solutions:
\begin{eqnarray*}
z&=&(e^{-j\pi})^{1/3}\;\;\;(e^{-j3\pi})^{1/3}\;\;\;(e^{-j5\pi})^{1/3}\\
z&=&(e^{-j\pi/3})\;\;\;(e^{-j\pi})\;\;\;(e^{-j5\pi/3})\\
\end{eqnarray*}
\end{example}
%}{}

%\rule[0.01in]{6in}{0.01in}

\tryit{\section{Example Solutions}

\noindent Example \ref{ex:rectopolar1}: Solution:
\begin{enumerate}
\item $1 \angle \pi/3$
\[
1 \angle \pi/3 =\cos(\pi/3)+j\sin(\pi/3)=\frac{1}{2}+\frac{\sqrt{3}}{2}j
\]
\item $2 \angle \pi$
\[
2 \angle \pi=2\cos(\pi)+j2\sin(\pi)=-2
\]
\end{enumerate}


\noindent Example \ref{ex:rectopolar2}: Solution:
\begin{enumerate}
\item $1+j$
\begin{eqnarray*}
r&=&\sqrt{1+1}=\sqrt{2}\\
\theta&=&\tan^{-1}(1)=\pi/4
\end{eqnarray*}
\item $-1+j$
\begin{eqnarray*}
r&=&\sqrt{1+1}=\sqrt{2}\\
\theta&=&\pi-\tan^{-1}(1)=3\pi/4
\end{eqnarray*}
\end{enumerate}

\noindent Example \ref{ex:complexalgebra1}: Solution:
\begin{enumerate}
\item 
\[
s_{1}+s_{1}^{*} = (1+j) + (1-j) = 2
\]
\item Since we are subtracting, we need to convert both numbers to rectangular form. Note that 
\[
s_{2} = e^{j\pi/3} = \cos(\pi/3) + j \sin(\pi/3) = \frac{1}{2} + j\frac{\sqrt{3}}{2}
\]
Thus,
\[
s_{1} - s_{2} = (1+j) - \left(\frac{1}{2} + j\frac{\sqrt{3}}{2}\right) = \frac{1}{2} + j\frac{2-\sqrt{3}}{2}
\]
\end{enumerate}

\noindent Example \ref{ex:euler1}: \input{examplesolutions/euler1}

\noindent Example \ref{ex:euler2}: Solution: 
\[
j^{j} = (e^{j\pi/2})^{j} = (e^{j(j\pi/2)}) = e^{-\pi/2} 
\]

\noindent Example \ref{ex:polar1}: First, note that
\[
(1+j) = \sqrt{2}e^{j\pi/4}
\]
Thus
\[
(1+j)2e^{j\pi/2} = \sqrt{2}e^{j\pi/4}2e^{j\pi/2} = 2\sqrt{2} e^{j3\pi/4}
\]



}{}
\end{document}


