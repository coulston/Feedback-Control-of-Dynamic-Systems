(a) We can use either Routh-Hurwitz or the quadratic formula to solve this part.  In the Routh case,
\[
\begin{array}{r|c c}
	s^2 & C & \lambda \\
	s^1 & \beta_1 & \\
	s^0 & \lambda & \\
\end{array}
\]
Since both $C$ and $\lambda$ are positive, we know that any positive value of $\beta_1$ will result in closed-loop stability.  Thus, $0<\beta_1<\infty$.

(b) Compare $G(s)$ to our standard second-order transfer function format:
\[
K\frac{\omega_n^2}{s^s+2 \zeta s\omega_n s + \omega_n^2} = \left(\frac{1}{\lambda}\right) \frac{\frac{\lambda}{C}}{s^2 + \frac{\beta_1}{C} s + \frac{\lambda}{C}}
\]
We know that in general settling time is determined from $t_s=\frac{4.6}{\zeta \omega_n}$, and therefore in this case $t_s = \frac{4.6 C}{\beta_1}$.  In the end, we conclude that the range of possible settling times goes from near zero (when $\beta_1=\infty$) to near infinity (when $\beta_1=0$.  This result helps to illustrate that a system can be BIBO stable even with an infinite range of step response specification results.

(c) There are an infinite number of possible answers to this question.  For example:
\begin{itemize}
	\item An extremely short settling time might lead to social or political stability problems because of too-rapid changes (assuming that a very short settling time would also imply a very short rise time).
	\item Similarly, an extremely short settling time could damage ecological systems.
	\item An extremely long settling time might result in failure to achieve a steady-state value in the end if human societies lose motivation to stick with required changes during the transition period.
\end{itemize}
