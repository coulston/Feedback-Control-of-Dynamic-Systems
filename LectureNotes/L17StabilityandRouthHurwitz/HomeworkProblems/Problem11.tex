Assume that the transfer function from radiative forcing $F(s)$ to surface temperature $T(s)$ on Earth can be written as
\[
G(s)=\frac{1}{Cs^2 + \beta_1s + \lambda}
\]
where $C$ is the surface layer heat capacity per unit area, $\beta_1$ is a function of thermal diffusivity, density, specific heat capacity, and thermal diffusivity.\footnote{Note that this is a not-quite-correct re-formulation of the equation in \\ 
D. MacMartin, B. Kravitz, D. Keith, and A. Jarvis, ``Dynamics of the coupled human-climate system resulting from closed-loop control of solar geoengineering,'' \textit{Climate Dynamics}, Vol. 43, pp. 243-258, 2014.\\
made to make the material more accessible to the tools learned so far in the class.}    Let $C=3.2 \times 10^6 Jm^{-2}K^{-1}$ and $\lambda = 1.2 Wm^{-2}K^{-1}$.

\begin{enumerate}
	\item Find the allowable range of $\beta_1$ for BIBO stability (don't worry about units).
	\item Compare your settling time for a $\beta_1$ very near the maximim and minimum values of the allowable BIBO stable ranges
	\item List 2 possible social justice implications of geo-engineering decisions that would result in $\beta_1$ values near the top and bottom ends of the allowable ranges.  Your answers may be brief and can be positive or negative, but must demonstrate an understanding of the link between an engineering decision (allowable size of parameter $\beta_1$) and social justice. 
\end{enumerate}

