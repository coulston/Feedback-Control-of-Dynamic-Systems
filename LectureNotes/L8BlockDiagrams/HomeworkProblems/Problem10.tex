Consider the problem of active prosthetic control, where a person missing a limb such as a hand uses his or her eyes and brain in a feedback loop to control the position of a prosthetic hand's finger. 

\begin{center}
\includegraphics[width=5in]{\mainfolder/LectureNotes/\lecturefolder/HomeworkProblems/Problem10fig.pdf}\footnote{Thanks to Anton Filatov, ME, CSM for the graphics.}
\end{center}

The block diagram below represents this feedback loop symbolically using transfer functions $C, G_1, G_2, G_3$ and $H$ to represent individual blocks. Use block diagram manipulation techniques to find the transfer function from the desired finger angular position $\theta_{des}$ to the actual finger angular position $\theta$.

\begin{center}
\includegraphics[width=5in]{\mainfolder/LectureNotes/\lecturefolder/HomeworkProblems/Problem10fig2.pdf}
\end{center}
