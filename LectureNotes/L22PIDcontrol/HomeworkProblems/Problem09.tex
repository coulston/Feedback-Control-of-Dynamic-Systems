% Note: this problem has many similarities to Problem08, but has been revised to consider a social justice perspective.
A remote village has won a grant to install a tank that can provide household water during the dry season.  To ensure a balance between protecting the pump (extending its lifetime as long as possible), ensuring enough water for the village's needs, and not wasting water by overfilling the tank, you are asked to design a Proportional/Integral (PI) control system to control the level of liquid in the tank.  Your control input is the volumetric flow input ${q}_{in}$ into the tank, which is shown in the figure below. The tank has an area of $100$ m$^2$, the density of the water is $\rho = 1000$ kg m$^{-2}$, and the flow resistance at the outlet valve is $400$  kg m$^{-4}$s$^{-1}$. \\

\begin{minipage}{6 in}
\begin{center}
\begin{tikzpicture}
\draw (.75,0) node[above] (tank) {\input{\mainfolder/DrawingElements/FluidElements/tank.tex}};
\draw[decorate,decoration={coil,aspect=0,segment length=5.85pt}] (-.45,2.25) -- ++(2.38,0);
\draw (-1.15,1) node (pipe1) {\input{\mainfolder/DrawingElements/FluidElements/pipe.tex}};
\draw (2.65,1) node (pipe2) {\input{\mainfolder/DrawingElements/FluidElements/valve.tex}};
\draw[->] (.2,.75) -- node[pos=.5,left] {$h$} ++(0,1.4);
\draw (tank.-90) node{Tank Area: $A$};
\draw (pipe2.90) node[above] {$R$};
\draw (.75,.8) node[above] {$p_{1}$};
\draw[<-] (pipe1.180) ++(.5,0) --  ++(-.5,0) node[left] {$q_{in}$};
\draw[->] (pipe2.0) ++(-.5,0) --  ++(.5,0) node[right] {$q_{out}$};
\draw (.75,2.5) node[above] {$p_{a}$};
\draw (pipe2.0) ++(1.5,0) node {$p_{a}$};
\end{tikzpicture}
\end{center}
\end{minipage}

\begin{enumerate}[(a)]
\item  Find the transfer function for the tank from input $q_{in}$ to output $h$. 
\item  Select three design specifications for your closed-loop controller + tank system and justify them briefly in a sentence or two.  At least one specification must be on a transient response parameter (rise time, settling time, or percent overshoot) and at least one must be on a steady-state characteristic (such as steady state error or disturbance rejection).  Your third specification may be from transient response, steady-state response, or something else you think is important.
\item  Draw a block diagram that represents the PI feedback control system. Include the reference signal for the desired height $H_{d}(s)$.  If one of your design specifications is related to disturbance rejection, you must include a disturbance.
\item  Determine the acceptable damping ratio and undamped natural frequency of the closed loop poles achieve the desired closed loop specifications.
\item  Find the gains for your controller that achieve the closed-loop pole locations to meet your design specifications.
\end{enumerate}
