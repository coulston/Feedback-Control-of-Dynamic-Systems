% Note: this problem is specific to the social justice research for fall 2014 and fall 2015 semesters.
\newcommand{\hwpath}[3]{\input{\mainfolder/LectureNotes/#1#2/HomeworkProblems/#2#3.tex}}

There are many possible answers encompassing a range of positive and negative effects.  Here are some examples:
\small
\begin{enumerate}[(i)]
%\item \textbf{Contextual Listening} Recall that contextual listening is a ``mechanism for identifying client and community needs, desires, and capacities.'' The fact that the specifications are not provided in the problem statement should prompt you as the engineer to consider how best to determine them.  Ideally, in a real world situation this would lead you to use contextual listening to seek the relevant information from the community, and formulate the specifications in partnership with community members.  
%\item \textbf{Identifying Structural Conditions} This criterion asks ``What economic, cultural, and other structural conditions gave rise to the needs that a client and/or community experiences?''  Being able to identify such conditions should also help you to determine the specifications, both in terms of numerical values and in terms of which specifications are required.  For example, if you learn that rainfall is infrequent but tends to be very intense when it falls (a structural condition), you may want to model rainfall events as impulsive disturbances and design a controller capable of disturbance rejection.  Similarly, you may learn that only certain people in the area are considered suitable to perform pump maintenance (a cultural condition), in which case you may choose to design a strategy that doesn't demand as much pump effort so that less frequent maintenance is required.
%\item \textbf{Acknowledging Political Agency/Mobilizing Power} ``When this design, model, or system is complete, who benefits?  Who does not?  Who suffers?  How do we acknowledge any imbalances between who benefits and who suffers?''  For this criterion, we may be prompted to question the very problem statement itself.  The problem states that the village won a grant to fund the tank.  Was the proposal (for the grant) a community effort, or was it written by someone with a particular agenda?  Does the granting agency have its own agenda (e.g., is it a company that builds tanks vs. an aid agency that awards money from which the community can select its own solution)?  These may not be questions you can answer (and certainly not within the information provided in the assigned problem), but it is worthwhile to start questioning problem statements in general.
\item \textbf{Increasing Opportunities and Resources} ``How can a given engineering design, model, or system increase opportunities and resources? Who benefits from such opportunities and resources? Who
does not? Who might suffer?'' This criterion might also lead you to question the problem statement.  It is given that the village has won a grant to install a tank, but is that even the best solution for the community?  What will happen to the entrepreneur who has built a business digging ditches for agriculture?  Is the community benefit from the tank worth his/her loss of income?  Even within the problem statement, are there specifications that you would choose that would favor some community members (e.g., the pump maintenance person) over others (e.g., the farmer)?
\item \textbf{Reducing Imposed Risks and Harms} ``How can a given engineering design, model, or system reduce imposed risked and harms? Who benefits from such risks and harms? Who does not?
Who might suffer? ''  Your controller design may be a key to the safety and reliability of the tank system, so it's imperative that you design carefully.
%\item \textbf{Enhancing Human Capabilities} The ten elements of this criterion include ``(1) life, (2) bodily health, (3), bodily integrity, (4) senses, imagination, and thought, (5) emotions, (6) practical reason, (7) affiliation, (8) other species, (9) play, and (10) control over one's political and material environment.''  Certainly you could imagine many possible benefits from a tank system - reliable water can be crucial for health and even life.  It can buffer people against impacts of climate change, giving them more control.  Changing an agricultural environment may also impact other species; some in positive, and others in potentially negative ways.  
\end{enumerate}

