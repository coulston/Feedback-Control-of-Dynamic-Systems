\begin{enumerate}[(a)]
\item \rule{0pt}{0pt}\\
 \includegraphics[width=6in]{\mainfolder/LectureNotes/\lecturefolder/HomeworkProblems/Problem07soln1.pdf}
\item \rule{0pt}{0pt}\\
 \includegraphics[width=6in]{\mainfolder/LectureNotes/\lecturefolder/HomeworkProblems/Problem07soln2.pdf}
\item\rule{0pt}{0pt}\\
 \includegraphics[width=6in]{\mainfolder/LectureNotes/\lecturefolder/HomeworkProblems/Problem07soln3.pdf}
\item\rule{0pt}{0pt}\\
 \includegraphics[width=6in]{\mainfolder/LectureNotes/\lecturefolder/HomeworkProblems/Problem07soln4.pdf}
 \item \rule{0pt}{0pt}\\
 \begin{center}
 \includegraphics[width=4in]{\mainfolder/LectureNotes/\lecturefolder/HomeworkProblems/Problem07/blockdiag}
 \end{center}
\begin{alltt}
plot(simout.time,simout.signals.values,'linewidth',2)
xlabel('Time')
ylabel('Temperature')
print -f1 -dpdf plotresults
\% Note that although the steady state error and settling time
\% specifications are met, the overshoot is over 20\%. This is 
\% due to the zero at -Ki/Kp = -1.08, which is close to the closed
\% loop poles that have real part -.46
\end{alltt}
\begin{center}
\includegraphics[width=4in]{\mainfolder/LectureNotes/\lecturefolder/HomeworkProblems/Problem07/plotresults}
\end{center}
 \end{enumerate}