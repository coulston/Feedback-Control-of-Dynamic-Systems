You are going to design a PD controller for a hydraulic system with mass load. The PD control configuration is shown below. The linearized valve equation is  $\delta q = 2 \delta x - 4 \delta p$. \vspace{.1in}
\begin{center}
\begin{tikzpicture}[sysblock/.style={draw,rectangle,inner sep=2pt,minimum width=1cm,minimum height=1cm,very thick}]


% tank
\draw[very thick] (0,0) -- ++(2,0) -- ++(0,-.5) -- ++(1,.5) -- ++(2,0) -- ++(0,-3) -- ++(-2,0) -- ++(-1,.5) -- ++(0,-.5) -- ++(-2,0);
\draw[very thick] (0,-.5) -- ++(1.5,0) -- ++(0,-.75) -- ++(-1.5,0);
\draw[very thick] (0,-1.75) -- ++(1.5,0) -- ++(0,-.75) -- ++(-1.5,0);
\draw[very thick] (2,-1) -- ++(1,.5) -- ++(0,-2) -- ++(-1,.5) -- cycle;
\draw (4,-3.5) node {\small Piston area: $1$};
\draw (0,-.25) node[left] {\small drain ($p_{0}$)};
\draw (0,-1.5) node[left] {\small supply ($p_{s}$)};
\draw (0,-2.75) node[left] {\small drain ($p_{0}$)};
\draw[->] (3,-.25) node[right] {\scriptsize$q$} -- ++(-.5,-.25);
\draw[->] (2.5,-2.5) -- ++(.5,-.25)  node[right] {\scriptsize$q$};
\draw (5.3,-1.9) node {\scriptsize $+$} ++(0,.4) node{\small $p$} ++(0,.4) node {\scriptsize$-$};

% valves
\draw[very thick,color=black] (1.75,-2.1) -- ++(0,3.1);
\draw[fill,left color=black,right color=black,middle color=white] (1.5,-.9) rectangle ++(.5,.35);
\draw[fill,left color=black,right color=black,middle color=white] (1.5,-2.4) rectangle ++(.5,.35);
\draw[|->] (1.5,1) node[left] (x) {$\delta x$} -- ++(0,-.5); 

% piston
\draw[fill,left color=black,right color=black,middle color=white] (4,-1.5) ++(-.2,0) rectangle ++(.4,2.5);
\draw[fill,left color=black,right color=black,middle color=white] (3,-1.75) rectangle ++(2,.5);
\draw[|->] (5.2,1) node[right] (y) {$\delta y$} -- ++(0,.5);

% load
%\draw (4,1) node[draw,very thick,above,dotted,rectangle,minimum width=.65in,minimum height=.4in] {Load};
\draw (4,1) node[above,inner sep=0,outer sep=0] {\input{\mainfolder/DrawingElements/MechanicalElements/mass2.tex}};
\draw (4,2) node {$m=1$};


% control
\draw (-2,1) node[sysblock]  (K) {$K_{p}$};
\draw (-2,2.5) node[sysblock]  (Kd) {$K_{d}s$};
\draw (0,1) node[draw,circle,inner sep=0pt,outer sep=0pt,very thick] (sum2) {\rule{12pt}{0pt}};
\draw (-3.5,1) node[draw,circle,inner sep=0pt,outer sep=0pt,very thick] (sum) {\rule{12pt}{0pt}};
\draw[thick,dotted,->] (y) -- ++(1,0) -- ++(0,3) -| (sum.90) node[above right=0pt] {$-$};
\draw[thick,dotted,->] (K.0) -- (sum2.180) node[above left] {$+$};
\draw[thick,dotted,->] (sum.90 |- Kd.180) -- (Kd.180); 
\draw[thick,dotted,->] (Kd.0) -| (sum2.90) node[above right] {$-$};
\draw[thick,dotted,->] (sum2.0) -- (x);
\draw[thick,dotted,->] (sum.0) -- (K.180);
\draw[<-,thick,dotted] (sum.180) node[above left=0pt] {$+$} -- ++(-1,0) node[above] {$\delta y_{d}$};
\end{tikzpicture}
\end{center}
\begin{enumerate}
\item Find the transfer function from $\delta x$ to $\delta y$. 
\item Re-draw the block diagram replacing the picture of the hydraulic system with the transfer function from part (a).
\item Find $K_{d}$ and $K_{p}$ such that the step response overshoot is 5\% and the settling time is 1 s.
\item Find the closed loop transfer function ($\frac{\delta Y(s)}{\delta Y_{d}(s)}$) for your choice of $K_{d}$ and $K_{p}$, and use \textsc{Matlab} to plot the step response.
\end{enumerate}
