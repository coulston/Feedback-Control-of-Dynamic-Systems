\mode<article>{\textsc{Matlab} has a graphical interface to run simulations of systems that are represented by block diagrams. While this is not as useful for creating simulation models of interconnected physical systems (in this case, another product, called Simscape is more useful) it is useful for creating simulations of systems with well defined input/output relationships, such as feedback control systems.

In addition to the example below, you can find a video introduction to Simulink at this link: \\ \hspace{.25in}\url{http://www.mathworks.com/videos/introduction-to-simulink-81623.html}\vspace{.1in}

You can start Simulink from the \textsc{Matlab} command line:\vspace{.1in}\\
\texttt{>> simulink} \vspace{.1in}\\
This will open up the following window, which give you access to all of the block elements in the Simulink Library}

\begin{frame}{Simulink Library}
\begin{center}
\begin{tikzpicture}
\draw (0,0) node {\includegraphics[height=2.8in]{Graphics/Simulink1}};
\draw (-4.15,3) node[circle,draw=red,very thick] {\rule{0pt}{6pt}};
\end{tikzpicture}
\end{center}
\end{frame}

To create a block diagram, we need to open up a new model window. This can be done by clicking on the new model icon (circled in red above). The result is a window that looks like the following:
\begin{frame}{Blank Simulink Model}
\begin{center}
\includegraphics[height=2.8in]{Graphics/Simulink2}
\end{center}
\end{frame}


\subsection{Simulink Example: PI Controller
\label{sec:simulinkPI}}
As an example, we will simulate the feedback control system for Example \ref{examp:PI}. We will grab blocks from the Simulink library and drag them over to the Simulink Model. The first block we need is a transfer function. There are two ways we can find a block. If we know which library it is in, we can click on that library in the ``Libraries'' window. So, for example, I know that transfer function blocks can be found in the ``Continuous''  library, and clicking on the word Continuous shows the blocks in that library. Note that the transfer function block is in the third column and third row.
\begin{frame}
\begin{center}
\includegraphics[height=2.8in]{Graphics/Simulink3}
\end{center}
\end{frame}

Alternately, we can search for the block using the search function. Type ``transfer'' into the search area (circled below) and you should see the list of blocks shown below. Note that the transfer function block is in the third column and first row.
\begin{frame}
\begin{center}
\begin{tikzpicture}
\draw (0,0) node {\includegraphics[height=2.8in]{Graphics/Simulink4}};
\draw (-2,2.95) node[ellipse,draw=red,minimum width=.6in,minimum height=.1in] {};
\end{tikzpicture}
\end{center}
\end{frame}

You can click on the transfer function block and drag it over to your model window. The model window should look like the following:
\begin{frame}
\begin{center}
\includegraphics[height=2.8in]{Graphics/Simulink5};
\end{center}
\end{frame}

Now do the following: In the library browser, click on the Continuous library, and drag the PID block to the model window, click on the Math Operations library and drag the sum to the model window, click on the Sources library and drag the step to the model window, and click on the Sinks library and drag the workspace block to the model window. Arrange the blocks is a line similar to the following:
\begin{frame}
\begin{center}
\includegraphics[height=2.8in]{Graphics/Simulink6};
\end{center}
\end{frame}

We can now connect up the blocks. Hover over the right side of the step block, in the region of the thick arrow head. The cursor should change to a plus sign. Click and drag the cursor over to the left side of the summer. An arrow should now connect these two blocks. Repeat this process until the model window is as follows: 
\begin{frame}
\begin{center}
\includegraphics[height=2.8in]{Graphics/Simulink7};
\end{center}
\end{frame}

We still need to connect the output of the transfer function block to the lower input of the summer. Do this by holding down the control key while clicking on the arrow that goes between the transfer function and the workspace block. Drag the cursor down and to the left, and then up until you hit the bottom of the summer. The result should look like the following:
\begin{frame}
\begin{center}
\includegraphics[height=2.8in]{Graphics/Simulink8};
\end{center}
\end{frame}

Now we can enter in the data for the blocks. Double click on the PID block, and a parameter window should open. Enter in the data so that the parameter window looks like the following: (Note that we are using $K_{p}$ and $K_{I}$ from Example 4). Now hit OK.
\begin{frame}
\begin{center}
\includegraphics[height=3.5in]{Graphics/Simulink9};
\end{center}
\end{frame}

Now double click on the transfer function block, and enter in the system transfer function for Example 4. Hit OK.
\begin{frame}
\begin{center}
\includegraphics[height=2.8in]{Graphics/Simulink10};
\end{center}
\end{frame}

The last thing we need to do is change the summer so that the bottom input is subtracted, not added. Double click on the summer, and change the last plus sign to a minus, as below. Hit OK.
\begin{frame}
\begin{center}
\includegraphics[height=2.8in]{Graphics/Simulink11};
\end{center}
\end{frame}

We are now ready to run the simulation. To choose the length of the simulation, enter the number of seconds in the box circled in red below. I have chosen 2 seconds. Then click on the green run button that is boxed in red below.
\begin{frame}
\begin{center}
\begin{tikzpicture}
\draw (0,0) node {\includegraphics[height=2.8in]{Graphics/Simulink12}};
\draw (3.2,2.6) node[ellipse,draw=red,minimum width=1.2in,minimum height=.1in] {};
\draw (-0.27,2.6) node[rectangle,draw=red,minimum width=.2in,minimum height=.2in] {};
\end{tikzpicture}
\end{center}
\end{frame}

Because we added a To Workspace block, the results of the simulation are in the variable \texttt{simout} in the Matlab workspace. The default is for this variable to be a \texttt{timeseries}, which is a structured variable that has several elements. Type \texttt{>> simout} at the command line to see the structure. We can plot the results either by typing \texttt{>> plot(simout)} or by typing \texttt{>> plot(simout.Time,simout.Data)}. This should result in the following plot:
\begin{frame}
\begin{center}
\includegraphics[height=2.8in]{Graphics/Simulink13};
\end{center}
\end{frame}

%\subsection{Simulink Example: PD Controller}
%
%For this example, we will implement a PD controller in Simulink. Let's take the block diagram for the PI controller in the previous example as our starting point. You might want to save this block diagram with a different name. 
%
%First, copy and paste the summer and PID control blocks, and connect them up as shown below. You can use command/control-I to flip a block direction, or right-click on the block and select rotate \& flip. 
%
%\begin{frame}
%\begin{center}
%\includegraphics[height=2.8in]{Graphics/PDsimulink1}
%\end{center}
%\end{frame}
%
%Click on the Transfer Fcn block and enter in the data for the motor system that is to be controlled.
%
%\begin{frame}
%\begin{center}
%\includegraphics[height=2.8in]{Graphics/PDsimulink2}
%\end{center}
%\end{frame}
%
%Click on the PID Controller block that will implement the proportional part of the controller, and enter in the proportional gain. (Actually, a simple gain block could also be used for this purpose.)
%
%\begin{frame}
%\begin{center}
%\includegraphics[height=3.5in]{Graphics/PDsimulink3}
%\end{center}
%\end{frame}
%
%Click on the PID Controller1 block that will implement the derivative part of the controller. Here we enter in the derivative gain, but we also have to choose a value for the filter coefficient \textsf{N}. We will select this to be 10 times the desired closed loop natural frequency. Since the desired closed loop $\omega_{n}$ is 22, enter 220 here.
%
%\begin{frame}
%\begin{center}
%\includegraphics[height=3.5in]{Graphics/PDsimulink4}
%\end{center}
%\end{frame}
%
%You can now run the simulation using the green arrow to start. 
%
%\begin{frame}
%\begin{center}
%\includegraphics[height=2.8in]{Graphics/PDsimulink5}
%\end{center}
%\end{frame}
%
%Type \texttt{>> plot(simout)} to get the following plot.
%
%\begin{frame}
%\begin{center}
%\includegraphics[height=2.8in]{Graphics/PDsimulink6}
%\end{center}
%\end{frame}
%
%Note that the rise time is 0.1s and the overshoot is 10\%, as desired.
%
%\textcolor{red}{Subsection ``What does the filter coefficient $N$ do?'' was moved to the PD control design lecture (\#13 in Fa'22)}