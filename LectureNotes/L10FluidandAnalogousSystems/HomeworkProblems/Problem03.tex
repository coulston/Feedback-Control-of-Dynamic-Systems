The Panama Canal uses a system of locks to raise and lower boats to the level of each section of the canal. Suppose that the boat pictured is in an enclosed lock that is 320 meters long and 33.5 meters wide, and the next lock is the same size. A valve is opened that connects the two locks (but no pumps are engaged), and water passes through the valve according to a linear valve relationship $Rq=(p_{1}-p_{2})$. The density of water is $\rho=1000$ kg$\cdot$m$^{-3}$ and the gravitational constant is $g=9.81$ m/s$^2$.
\begin{center}
\includegraphics[width=3.8in]{\mainfolder/LectureNotes/\lecturefolder/HomeworkProblems/Problem03/Panama_Canal_Gatun_Locks.jpg}\\
{\tiny Source: Stan Shebs via Wikimedia commons}
\end{center}
\begin{enumerate}
\item Draw a diagram with ideal elements that models this system
\item Find a differential equation and initial conditions that describes the behavior of the system (assume the time the valve is opened is $t=0$).
\item Suppose the water level of the filled lock is 20 meters higher than the empty lock, and it takes 1 hour for the water level difference to shrink by 10 meters. What is the value of $R$? 
\end{enumerate}
