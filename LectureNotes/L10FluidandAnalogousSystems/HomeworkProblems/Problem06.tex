Water reservoirs can be used to protect communities in case of flooding. In the flood that struck Colorado in September 2013, many reservoirs filled above capacity very quickly, flooding downstream communities, killing eight people, disrupting thousands, and causing billions of dollars in damage. The images below show Barker Reservoir (in Nederland, left), which flows into Boulder Creek, and an aerial view of a flooded neighborhood in Longmont (right). (Note: the fact that Barker Reservoir overflowed is not the cause of the Longmont flooding, though both were caused by the same rain event.)

\begin{center}
\includegraphics[width=4.8in]{\mainfolder/LectureNotes/\lecturefolder/HomeworkProblems/Problem06image12.png}\\
{\tiny Barker Reservoir (left; \url{http://us.geoview.info/nederland_and_barker_reservoir,3269496p}) on a sunny day and Longmont (right; \url{http://www.livescience.com/39635-colorado-flood-photos.html}) during the 2013 Colorado flood. The spillway that releases water from Barker Reservoir when it reaches capacity is visible to the right side of the dam in the lower center of the picture.}
\end{center}

If the inflow to the reservoir is approximated by a scaled step input $q_{in} = \alpha u(t)$, calculate how fast the rainwater must flow into Barker Reservoir to cause it to overflow and release floodwaters down into Boulder. \textbf{That is, find $\alpha$ in $m^3/s$}. Consider the following:

\begin{itemize}
	\item Barker Reservoir has a maximum capacity $V_{max} = 258,000,000 m^3$ of water.
	\item Assume that the reservoir was filled to 80\% capacity prior to the start of the rain.
	\item Assume that the (constant) discharge rate into Boulder Creeks is typical for September; that is, $q_{out} = 2 m^3/s$ until the reservoir reaches maximum capacity and starts to overflow.
	\item Assume that, once the rain begins, the reservoir takes 48 hours to fill to its maximum capacity $V_{max}$.
\end{itemize}
