A hydraulic system is used to move a large mass. A sensor monitors the position of the mass and a feedback control system is implemented by setting the valve position as $x=10(y_{d}-y)$ where $y_{d}$ is the desired position of the mass. The dotted lines indicate the information flow that implements this controller. The linearized servo valve equation is $\delta q =  \delta x - 5 \delta p$. Find the transfer function $\delta Y(s)/\delta Y_{d}(s)$, and the step response settling time.
\begin{center}
\resizebox{!}{3.5in}{
\begin{tikzpicture}[sysblock/.style={draw,rectangle,inner sep=2pt,minimum width=1cm,minimum height=1cm,very thick}]


% tank
\draw[very thick] (0,0) -- ++(2,0) -- ++(0,-.5) -- ++(1,.5) -- ++(2,0) -- ++(0,-3) -- ++(-2,0) -- ++(-1,.5) -- ++(0,-.5) -- ++(-2,0);
\draw[very thick] (0,-.5) -- ++(1.5,0) -- ++(0,-.75) -- ++(-1.5,0);
\draw[very thick] (0,-1.75) -- ++(1.5,0) -- ++(0,-.75) -- ++(-1.5,0);
\draw[very thick] (2,-1) -- ++(1,.5) -- ++(0,-2) -- ++(-1,.5) -- cycle;
\draw (4,-3.5) node {\small Piston area: $1$};
\draw (0,-.25) node[left] {\small drain ($p_{0}$)};
\draw (0,-1.5) node[left] {\small supply ($p_{s}$)};
\draw (0,-2.75) node[left] {\small drain ($p_{0}$)};
\draw[->] (3,-.25) node[right] {\scriptsize$q$} -- ++(-.5,-.25);
\draw[->] (2.5,-2.5) -- ++(.5,-.25)  node[right] {\scriptsize$q$};
\draw (5.3,-1.9) node {\scriptsize $+$} ++(0,.4) node{\small $p$} ++(0,.4) node {\scriptsize$-$};

% valves
\draw[very thick,color=black] (1.75,-2.1) -- ++(0,3.1);
\draw[fill,left color=black,right color=black,middle color=white] (1.5,-.9) rectangle ++(.5,.35);
\draw[fill,left color=black,right color=black,middle color=white] (1.5,-2.4) rectangle ++(.5,.35);
\draw[|->] (1.5,1) node[left] (x) {$x$} -- ++(0,-.5); 

% piston
\draw[fill,left color=black,right color=black,middle color=white] (4,-1.5) ++(-.2,0) rectangle ++(.4,2.5);
\draw[fill,left color=black,right color=black,middle color=white] (3,-1.75) rectangle ++(2,.5);
\draw[|->] (5.2,1) node[right] (y) {$y$} -- ++(0,.5);

% load
%\draw (4,1) node[draw,very thick,above,dotted,rectangle,minimum width=.65in,minimum height=.4in] {Load};
\draw (4,4.55) node[rotate=-90]  {\input{\mainfolder/DrawingElements/MechanicalElements/ground.tex}};
\draw (4,3.55) node[rotate=90] (d) {\begin{tikzpicture}
\draw[very thick] (-.2,0) -- (0,0);
\draw (.75,0) node {\begin{tikzpicture}
\draw[very thick] (-.2,0) -- (0,0);
\draw (.75,0) node {\begin{tikzpicture}
\draw[very thick] (-.2,0) -- (0,0);
\draw (.75,0) node {\input{\mainfolder/DrawingElements/MechanicalElements/damper.tex}};
\draw (.75,0) node[above=9pt] {$b$};
\draw[very thick] (1.5,0) -- ++(.2,0);
    \draw[<-,thick] (1.5,0) ++(.2,0) -- ++(.5,0) node[right] {$f$};
    \draw[<-,thick] (-.2,0) -- ++(-.5,0) node[left] {$f$};
    \draw[|->,thick] (-.2,.4) node[above=2pt] {$x_{1}$} -- ++(.5,0);  
    \draw[|->,thick] (1.7,.4) node[above=2pt] {$x_{2}$} -- ++(.5,0);  
    \draw (.6,-.6) node {$x=x_{1}-x_{2}$};
  %  \draw (.6,-1.2) node {$f=b\dot{x}$};
\end{tikzpicture}};
\draw (.75,0) node[above=9pt] {$b$};
\draw[very thick] (1.5,0) -- ++(.2,0);
    \draw[<-,thick] (1.5,0) ++(.2,0) -- ++(.5,0) node[right] {$f$};
    \draw[<-,thick] (-.2,0) -- ++(-.5,0) node[left] {$f$};
    \draw[|->,thick] (-.2,.4) node[above=2pt] {$x_{1}$} -- ++(.5,0);  
    \draw[|->,thick] (1.7,.4) node[above=2pt] {$x_{2}$} -- ++(.5,0);  
    \draw (.6,-.6) node {$x=x_{1}-x_{2}$};
  %  \draw (.6,-1.2) node {$f=b\dot{x}$};
\end{tikzpicture}};
\draw (.75,0) node[above=9pt] {$b$};
\draw[very thick] (1.5,0) -- ++(.2,0);
    \draw[<-,thick] (1.5,0) ++(.2,0) -- ++(.5,0) node[right] {$f$};
    \draw[<-,thick] (-.2,0) -- ++(-.5,0) node[left] {$f$};
    \draw[|->,thick] (-.2,.4) node[above=2pt] {$x_{1}$} -- ++(.5,0);  
    \draw[|->,thick] (1.7,.4) node[above=2pt] {$x_{2}$} -- ++(.5,0);  
    \draw (.6,-.6) node {$x=x_{1}-x_{2}$};
  %  \draw (.6,-1.2) node {$f=b\dot{x}$};
\end{tikzpicture}};
\draw (d) node[right=16pt] {$b=1$};
\draw (4,1) node[above,inner sep=0,outer sep=0] {\input{\mainfolder/DrawingElements/MechanicalElements/mass2.tex}};
\draw (4,2) node {$m=1$};


% control
\draw (-.5,1) node[sysblock]  (K) {$10$};
\draw (-2,1) node[draw,circle,inner sep=0pt,outer sep=0pt,very thick] (sum) {\rule{12pt}{0pt}};
\draw[thick,dotted,->] (y) -- ++(0.5,0) -- ++(0,4) -| (sum.90) node[above right=0pt] {$-$};
\draw[thick,dotted,->] (K.0) -- (x);
\draw[thick,dotted,->] (sum.0) -- (K.180);
\draw[<-,thick,dotted] (sum.180) node[above left=0pt] {$+$} -- ++(-1,0) node[above] {$y_{d}$};
\end{tikzpicture}
}
\end{center}


