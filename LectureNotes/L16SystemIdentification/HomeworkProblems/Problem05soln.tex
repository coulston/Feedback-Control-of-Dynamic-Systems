(a) It would be reasonable to use the median rise time, $t_r = 12.05 s$, from which to compute $\sigma = \frac{2.2}{t_r} = 0.18$.

(b) Looking at a few of the step responses, we can describe the system's behavior qualitatively. For example:\\
\begin{itemize}
\item At time $t = 100$ s, a negative step pitch input $-3u(t)$ deg is applied, and the rotor power increases. This implies a negative gain (different ``directions'' for the input and output).
\item At time $t = 200$ s, a positive step pitch input $2u(t)$ deg is applied, and the rotor power also increases. This implies a positive gain.
\item At time $t = 300$ s, a positive step pitch input $2u(t)$ deg is applied, and the rotor power again decreases. This implies a negative gain.
\end{itemize}

Therefore, we conclude that we cannot determine the sign of the transfer function's DC gain $K$ from this series of experiments. (In fact, what is actually happening is that the relationship from blade pitch angle $\beta$ to power $P$ is nonlinear, so it's not actually possible to fully represent it using a single transfer function.)

(c) The Matlab commands\\
\ttfamily
>> sigma=0.18;\\
>> gs=tf(-sigma,[1 sigma])\\
>> step(gs)\\
\rmfamily

result in the plot

\begin{center}
\includegraphics[width=4.5in]{\mainfolder/LectureNotes/\lecturefolder/HomeworkProblems/Problem05soln1.pdf}\\
\end{center}

From the data markers, the rise time is $t_r \approx 12.3$ s as expected, and the negative gain $K$ results in the power decreasing as shown in the problem's figure. However, the magnitude of $K$ is clearly wrong, since this figure shows a change in amplitude of just 1 kW and the problem's figure shows a change in amplitude of approximately 20 kW.