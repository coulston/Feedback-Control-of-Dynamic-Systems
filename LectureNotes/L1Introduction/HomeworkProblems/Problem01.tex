In the introduction to this class, you learned that feedback control impacts many aspects of
our modern lives. You have also learned that we will be working with both signals
represented by arrows (or variables), and systems represented by block. For example, in
economics, the Law of Demand and Supply is based on the fact that the market demand for an
item decreases as its price increases, and the market supply usually increases as its price
increases. This concept can be represented as a feedback control system. The system is the
economic system with ``price stability'' as the input signal (R) and ``the actual market price'' as the
output signal (Y). For a non-complicated system the following four elements can be used to
represent the system: the Demander, the Pricer, the Supplier, and the Market where the item
is sold and bought.

\begin{center}
\includegraphics[width=4.5in]{\mainfolder/LectureNotes/\lecturefolder/HomeworkProblems/Problem01image.pdf}
\end{center}

Think of a system (e.g. technical, social, political, economic, biological), and then sketch
and label the signals (arrows) and system(s) (block(s)) involved. Make sure to label both. 
You may look in the news for ``feedback loop'' and describe something closed-loop
you read about, along with giving citation information (Author, Title, URL, date accessed, etc.). 
Note: Your feedback loop does not need to be as complex as the provided example but must provide at least one feedback path.