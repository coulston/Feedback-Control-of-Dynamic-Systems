
\mode<presentation>
{
  \usetheme{CambridgeUS}
  \usecolortheme{whale}
  \usecolortheme{lily}

  \setbeamercovered{transparent}
  \usefonttheme[onlymath]{serif}
}

\title[\SolvingDifferentialEquationsIIShortName] % (optional, use only with long paper titles)
{\course: \SolvingDifferentialEquationsIIName\license}

\subtitle
{Lecture \SolvingDifferentialEquationsIINumber} % (optional)




\begin{document}

\begin{frame}
  \titlepage
\end{frame}

\mode<article>{
\maketitle
\tableofcontents
}

%\mode<presentation>{
%\begin{frame}{Outline}
%  \tableofcontents
%  % You might wish to add the option [pausesections]
%\end{frame}}


\section{Pre-requisite Material}
This lecture assumes that the reader is familiar with the following material:
\begin{itemize}
\item Lecture \ModelingMechanicalSystemsNumber:~\ModelingMechanicalSystemsName
\item Lecture \ElectricalSystemsNumber:~\ElectricalSystemsName
\item Lecture \SolvingDifferentialEquationsINumber:~\SolvingDifferentialEquationsIName
\end{itemize}



\section{Inverse Laplace Transform with Repeated roots}
Since polynomials can have repeated roots, we need to determine how to take the inverse Laplace Transform when the denominator roots are repeated.

\subsection{Laplace Transform Pairs with repeated roots}
\begin{frame}{Laplace Transform Pairs with repeated roots}
Repeated roots occur when an exponential or sinusoid is multiplied by $t$ or $t^{n}$. First, let's derive the Laplace Transform of $t^{n}$. We already have established the Laplace Transform pair
\[
\step(t) \overset{\mathcal{L}}{\longleftrightarrow} \frac{1}{s},
\]
and the integration theorem
\[
\mathcal{L}\left\{ \int_{0}^{t}f(\tau )d\tau \right\} =%
\frac{1}{s}\mathcal{L}\left\{ f(t)\right\}. 
\]
\end{frame}
\begin{frame}{Laplace Transform Pairs with repeated roots}
Note that a ramp is the integral of a step. That is, the ramp function $t\step(t)$ can be defined via
\[
t\step(t) = \int_{0^{-}}^{t} \step(t) dt.
\]
Using the integration theorem gives us
\[
t\step(t) \overset{\mathcal{L}}{\longleftrightarrow} \frac{1}{s^{2}}.
\]
Thus, a ramp has two poles at $s=0$. 
\end{frame}
\begin{frame}{Higher Powers}
The Laplace Transform for higher powers of $t$ can be found via further integration. For example,
\[
\frac{1}{2}t^{2}\step(t) =  \int_{0^{-}}^{t} t\step(t) dt,
\]
which implies
\[
\frac{1}{2}t^{2}\step(t) \overset{\mathcal{L}}{\longleftrightarrow} \frac{1}{s^{3}}.
\]
and even higher orders of $t$ can be found similarly. What we see is that powers of $t$ give us a denominator with repeated roots at $s=0$. 
\end{frame}
Now, we can use the frequency shift theorem
\[
\mathcal{L}\left\{ e^{-s_{0}t}f(t)\right\}
=F(s+s_{0})
\]
to show that
\[
te^{-at} \step(t)  \overset{\mathcal{L}}{\longleftrightarrow} \frac{1}{(s+a)^{2}}.
\]
Thus the product of a ramp and exponential will give a repeated real root.

Repeated imaginary or complex roots occur when $t$ is multiplied by a sinusoid or decaying sinusoid. The following table is a partial list of the Laplace Transform pairs with repeated roots
\begin{center}
\begin{tabular}{l|l}
$f(t)$ & $F(s)$ \\\hline
$te^{-at}$ & $\frac{1}{(s+a)^{2}}$\\
$\frac{t^{n}e^{-at}}{n!}$ & $\frac{1}{(s+a)^{n+1}}$ \\
$t\sin(\omega t)$ & $\frac{2\omega s}{\left(s^{2} +\omega^{2}\right)^{2}}$ \\
$te^{-at}\sin(\omega t)$ & $\frac{2\omega(s+a)}{\left((s+a)^{2} +\omega^{2}\right)^{2}}$
\end{tabular}
\end{center}

\subsection{Partial Fraction Expansion with Repeated Roots}

Suppose we had the Laplace Transform
\[
X(s) = \frac{s+3}{(s^{2}+2s+1)(s+2)},
\]
and we want to find the inverse Laplace Transform. The denominator of this function has roots at $-1, -1$ and $-2$. Thus, the root at $-1$ is repeated. In order to apply the partial fraction expansion correctly, we need one term for each root:
\[
 \frac{s+3}{(s+1)^{2}(s+2)} = \frac{A}{s+1} + \frac{B}{(s+1)^{2}} + \frac{C}{s+2}
\]
Since we have expanded our Laplace Transform pairs to include repeated roots, once we find the residues, the inverse Laplace Transform is easy. Thus, we have the general rule:

\begin{center}
\fbox{\begin{minipage}{4in}If the term $(s+a)^{n}$ appears in the denominator, then the partial fraction expansion will include the terms $(s+a)$, $(s+a)^{2}$, $\cdots$, $(s+a)^{n}$.\end{minipage}}
\end{center}

Of course, we still have to solve for the residues $A$, $B$, $C$. The good news is that the residue formula will hold for the highest power, as well as all other terms. So we have
\begin{align*}
B & = \left.(s+1)^{2}\frac{s+3}{(s+1)^{2}(s+2)}\right|_{s=-1} = \frac{2}{1} = 2\\
C & = \left.(s+2)\frac{s+3}{(s+1)^{2}(s+2)}\right|_{s=-2} = \frac{1}{1} = 1
\end{align*}
However, the residue formula will {\em not} hold for $A$. There is an alternate residue formula, but it is usually easier to simply return to putting the right hand side over a common denominator, substituting in the values for $B$ and $C$ found above.
\begin{align*}
 \frac{s+3}{(s+1)^{2}(s+2)} &= \frac{A(s+1)(s+2) + 2(s+2) + 1(s+1)^2}{(s+1)^{2}(s+2)} \\
 & = \frac{As^{2}+3As+2A + 2s+4 + s^{2}+2s+1}{(s+1)^{2}(s+2)} 
\end{align*}
Solve for $A$ by equating coefficients for powers of $s$. For example, using the coefficients for $s^{2}$,
\[
0 = A +1
\] 
which implies $A=-1$. The final partial fraction expansion becomes
\[
X(s)= \frac{-1}{s+1} + \frac{2}{(s+1)^{2}} + \frac{1}{s+2}
\]
The inverse Laplace Transform is then
\[
x(t) = \left(-e^{-t} + 2te^{-t} + e^{-2t}\right)\step(t)
\]
\section{Differential Equation Examples}
Let's try some examples to solidify the process of solving differential equations using Laplace Transforms.

\subsection{Example 1}
\begin{frame}{Spring and Damper}
A spring with spring constant $k=4$ N/m and damper with damping coefficient $b=2$ Ns/m is connected in parallel to a wall. A force of $f_{in}=1$ N is applied for  $t\geq0$. If the initial displacement of the right side of the spring and damper is $x=1$ m at $t=0$, find $x(t)$ for $t\geq 0$
\begin{center}
\begin{tikzpicture}[inner sep=0pt,outer sep=0pt,very thick]
\draw (1,0) node (gnd) {\input{\mainfolder/DrawingElements/MechanicalElements/ground.tex}};
\draw (3.5,0.5) node (K1) {\begin{tikzpicture}
\draw (.75,0) node[inner sep=0,outer sep=0] (K1) {\begin{tikzpicture}
\draw (.75,0) node[inner sep=0,outer sep=0] (K1) {\input{\mainfolder/DrawingElements/MechanicalElements/spring.tex}};
\draw (K1)  node[above=6pt] {$k$};
\draw[very thick] (K1.180) -- ++(-.2,0);
\draw[very thick] (K1.0) -- ++(0.2,0);
\draw[<-,thick] (K1.0) ++(.2,0) -- ++(.5,0) node[right] {$f$};
\draw[<-,thick] (K1.180) ++(-.2,0) -- ++(-.5,0) node[left] {$f$};
\draw[|->,thick] (K1.180) ++(-.2,.4) node[above=2pt] {$x_{1}$} -- ++(.5,0);  
\draw[|->,thick] (K1.0) ++(.2,.4) node[above=2pt] {$x_{2}$} -- ++(.5,0);  
\draw<2-> (K1) ++(0,-.6) node {$f=k(x_{1}-x_{2})$};
\end{tikzpicture}
};
\draw (K1)  node[above=6pt] {$k$};
\draw[very thick] (K1.180) -- ++(-.2,0);
\draw[very thick] (K1.0) -- ++(0.2,0);
\draw[<-,thick] (K1.0) ++(.2,0) -- ++(.5,0) node[right] {$f$};
\draw[<-,thick] (K1.180) ++(-.2,0) -- ++(-.5,0) node[left] {$f$};
\draw[|->,thick] (K1.180) ++(-.2,.4) node[above=2pt] {$x_{1}$} -- ++(.5,0);  
\draw[|->,thick] (K1.0) ++(.2,.4) node[above=2pt] {$x_{2}$} -- ++(.5,0);  
\draw<2-> (K1) ++(0,-.6) node {$f=k(x_{1}-x_{2})$};
\end{tikzpicture}
};
\draw (K1) node[above=14pt] {$k$};
\draw (3.5,-0.5) node (B1) {\begin{tikzpicture}
\draw[very thick] (-.2,0) -- (0,0);
\draw (.75,0) node {\begin{tikzpicture}
\draw[very thick] (-.2,0) -- (0,0);
\draw (.75,0) node {\input{\mainfolder/DrawingElements/MechanicalElements/damper.tex}};
\draw (.75,0) node[above=9pt] {$b$};
\draw[very thick] (1.5,0) -- ++(.2,0);
    \draw[<-,thick] (1.5,0) ++(.2,0) -- ++(.5,0) node[right] {$f$};
    \draw[<-,thick] (-.2,0) -- ++(-.5,0) node[left] {$f$};
    \draw[|->,thick] (-.2,.4) node[above=2pt] {$x_{1}$} -- ++(.5,0);  
    \draw[|->,thick] (1.7,.4) node[above=2pt] {$x_{2}$} -- ++(.5,0);  
    \draw (.6,-.6) node {$x=x_{1}-x_{2}$};
  %  \draw (.6,-1.2) node {$f=b\dot{x}$};
\end{tikzpicture}};
\draw (.75,0) node[above=9pt] {$b$};
\draw[very thick] (1.5,0) -- ++(.2,0);
    \draw[<-,thick] (1.5,0) ++(.2,0) -- ++(.5,0) node[right] {$f$};
    \draw[<-,thick] (-.2,0) -- ++(-.5,0) node[left] {$f$};
    \draw[|->,thick] (-.2,.4) node[above=2pt] {$x_{1}$} -- ++(.5,0);  
    \draw[|->,thick] (1.7,.4) node[above=2pt] {$x_{2}$} -- ++(.5,0);  
    \draw (.6,-.6) node {$x=x_{1}-x_{2}$};
  %  \draw (.6,-1.2) node {$f=b\dot{x}$};
\end{tikzpicture}};
\draw (B1) node[below=14pt] {$b$};


\draw (K1.180) -- (K1.180 -| gnd.0);
\draw (B1.180) -- (B1.180 -| gnd.0);
\draw (K1.0) -- ++(0.5,0) |- ++(1,-.5) node[circle,fill,inner sep=0] (F) {\rule{0pt}{4pt}};
\draw (B1.0) -- ++(0.5,0) |- ++(0,.5);
\draw[|->,thick] (F) ++(0,.5) node[above=6pt] {$x$} -- ++(.5,0);
\draw[->,thick] (F) -- ++(.75,0) node[right=2pt] {$f_{in}$}; 

\end{tikzpicture}
\end{center}
\end{frame}
This system is governed by the differential equation
\[
b \dot{x} + k x = f_{in}  \\
\]
Thus, we can ask: what is the solution to the differential equation
\[
2 \dot{x} + 4 x = 1 \qquad x(0)=1 \\
\]
\tryit{
\begin{boxedminipage}{6.5 in}
\rule{0pt}{12pt}\vspace{2in}
\color{lightgray}
\[
x(t) = \left(\frac{1}{4} + \frac{3}{4}e^{-2t}\right)\step(t)
\]
\end{boxedminipage}\\
}{\begin{boxedminipage}{6.5in}
Take the Laplace Transform of both sides:
\[
2(sX(s) - 1) + 4X(s) = \frac{1}{s}
\]
Solve for X(s):
\begin{align*}
(2s+4)X(s) &= \frac{2s+1}{s} \\
X(s) = \frac{s +\frac{1}{2} }{s(s+2)}
\end{align*}
Find partial fraction expansion:
\begin{align*}
\frac{s +\frac{1}{2} }{s(s+2)} = \frac{A}{s} + \frac{B}{s+2}
\end{align*}
where
\begin{align*}
A &= \left.\frac{s +\frac{1}{2} }{s+2}\right|_{s=0} = \frac{1}{4}\\
B &= \left.\frac{s +\frac{1}{2}}{s}\right|_{s=-2} = \frac{3}{4}\\
\end{align*}
Giving inverse Laplace Transform
\[
x(t) = \left(\frac{1}{4} + \frac{3}{4}e^{-2t}\right)\step(t)
\]
\end{boxedminipage}
}

\subsection{Example 2}
\begin{frame}{Circuit Problem}
\begin{center}
An LRC circuit has applied voltage $v_{in}=1$ for $t\geq0$. If $v_{out}(t)$ was zero for $t\leq0$, find $v_{out}(t)$ for $t\geq 0$.\\
\input{figures/seriescircuit.tex}
\end{center}
\end{frame}

Previously, we found that this system was governed by the differential equation
\[
CL\ddot{v}_{out}+CR\dot{v}_{out} + v_{out} = v_{in}
\]
Thus, we can ask: what is the solution to the differential equation
\begin{equation*}
\frac{1}{4}\ddot{v}_{out}+\dot{v}_{out}+v_{out}=1\qquad v_{out}(0)=0,\text{ }\dot{v}_{out}(0)=0
\end{equation*}%
\tryit{
\begin{boxedminipage}{6.5 in}
\rule{0pt}{12pt}\vspace{2in}
\color{lightgray}
\[
v_{out}(t) = (1 - e^{-2t} -2te^{-2t})\step(t)
\]
\end{boxedminipage}\\
}{\begin{boxedminipage}{6.5in}
First, take the Laplace Transform of both sides (noting that initial conditions are zero):
\[
\frac{1}{4}s^{2}V_{out}(s) + sV_{out}(s) + V_{out}(s) = \frac{1}{s}.
\]
Then, solve for $V_{out}(s):$
\begin{align*}
V_{out}(s) &= \frac{4}{(s^{2}+4s+4)s} \\
 &= \frac{4}{s(s+2)^{2}} \\
\end{align*}
Find partial fraction expansion:
\[
V_{out}(s) = \frac{A}{s} +\frac{B}{s+2} + \frac{C}{(s+2)^{2}}
\]
Residues:
\begin{align*}
A &= \left.\frac{4}{(s+2)^{2}}\right|_{s=0} = 1\\
C & =\left.\frac{4}{s}\right|_{s=-2} = -2 
\end{align*}
Solve for remaining coefficients
\[
\frac{4}{s(s+2)^{2}} = \frac{(s^{2}+4s+4) + B(s^{2}+2s) - 2s}{s(s+2)^{2}} 
\]
Thus $B=-1$, and
\[
v_{out}(t) = (1 - e^{-2t} -2te^{-2t})\step(t)
\]
\end{boxedminipage}}

\section{Laplace Transform Table}

Here is a summary of the Laplace Transforms and other results derived in the last few lectures. 

\begin{center}
\resizebox{5in}{!}{\begin{tabular}{cc}\toprule
\multicolumn{2}{c}{Laplace transform pairs}\\\midrule
\hspace{1in}$f(t)$\hspace{1in} & \hspace{1in}$F(s)$\hspace{1in}  \\\midrule
\rule{0pt}{16pt}Unit impulse $\delta(t)$ & 1 \\[.15cm]\midrule
\rule{0pt}{16pt}Unit step $\step(t)$ & $\frac{1}{s}$ \\[.15cm]\midrule
\rule{0pt}{16pt}$t\step(t)$ & $\frac{1}{s^2}$ \\[.15cm]\midrule
\rule{0pt}{16pt}$\frac{1}{2}t^{2}\step(t)$ & $\frac{1}{s^{3}}$  \\[.15cm]\midrule
\rule{0pt}{16pt}$Ae^{at}\step(t)$ & $\frac{A}{s-a}$  \\[.15cm]\midrule
\rule{0pt}{16pt}$te^{at}\step(t)$ & $\frac{1}{(s-a)^{2}}$ \\[.15cm]\midrule
\rule{0pt}{16pt}$\frac{1}{2}t^2e^{at}\step(t)$ & $\frac{1}{(s-a)^{3}}$ \\[.15cm]\midrule
\rule{0pt}{16pt}$\sin(\omega t)\step(t)$ & $\frac{\omega}{s^2+\omega^2}$ \\[.15cm]\midrule
\rule{0pt}{16pt}$\cos(\omega t)\step(t)$ & $\frac{s}{s^2 + \omega^2}$ \\[.15cm]\midrule
\rule{0pt}{16pt}$e^{-at}\sin(\omega t)\step(t)$ & $\frac{\omega}{(s+a)^2+\omega^{2}}$ \\[.15cm]\midrule
\rule{0pt}{16pt}$e^{-at}\cos(\omega t)\step(t)$ & $\frac{s+a}{(s+a)^2+\omega^{2}}$ \\[.15cm]\midrule
\rule{0pt}{16pt}$\frac{df}{dt}$  & $sF(s) - f(0^{-})$ \\[.15cm]\midrule
\rule{0pt}{16pt}$\frac{d^{n}f}{dt^{n}}$ & $s^{n}F(s) - \sum_{k=1}^{n}s^{n-k}\frac{d^{k-1}f}{dt^{k-1}}(0^{-})$ \\[.15cm]\midrule
\rule{0pt}{16pt}$\int_{0}^{t} f(\tau)d\tau$ & $\frac{F(s)}{s}$ \\[.15cm]
\bottomrule
\end{tabular}}\vspace{.1in}
\end{center}


\section{Application Example} %new

A wind turbine is a rotational mechanical system that has a spinning ``rotor'' (blades plus their connections) with a large rotational inertia \textit{J}. Although most of the time the turbine is expected to spin continuously to produce electricity, occasionally it is desirable to use the generator in reverse (as a motor providing torque \(\tau \))to turn the rotor to a certain angular position \(\theta \) so that maintenance can be performed.

\begin{center}
	\includegraphics[height=2in]{figures/windturbine1.png}
	\includegraphics[height=2in]{figures/windturbine2.png}
\end{center}

The differential equation relating the input torque to the output position (angular acceleration and velocity) is given by
\begin{equation*}
	J\ddot{\theta} + b\dot{\theta} = \tau ,
\end{equation*}
where $b$ is the rotational damping coefficient. You decide to try to achieve a constant angular position \(\theta(t) = \theta_{setpoint}\) by applying a unit step torque input \(\tau = u(t), t \geq 0\), as shown in this block diagram. Is this a successful approach?
\begin{center}
	\includegraphics[width=3.5in]{figures/windturbineDiagram.png}
\end{center}

\textbf{\underline{Solution}}

Replace the torque $\tau$ with the unit step function and then take the Laplace transform of the equation:
\begin{equation*}
	\theta(s)(Js^2+bs)=\frac{1}{s}
\end{equation*}
Solve for \(\theta (s) \):
\begin{equation*}
	\theta(s)=\frac{1}{s(Js^2+bs)}=\frac{1}{s^2(Js+b)}
\end{equation*}
Use partial fraction expansion 
\begin{equation*}
	\frac{1}{s^2(Js+b)} = \frac{A}{s}+\frac{B}{s^2}+\frac{C}{Js+b}
\end{equation*}
and solve for the numerator coefficients by picking useful values of $s$:
\begin{equation*}
	1 = As(Js+b)+B(Js+b)+Cs^2
\end{equation*}
\begin{equation*}
	s=0: \indent  1=Bb \indent \Rightarrow B=\frac{1}{b}
\end{equation*}
\begin{equation*}
	s=\frac{-b}{J}: \indent  1=A(0)+B(0)+C \bigg( \frac{-b}{J}\bigg)^2  \indent \Rightarrow C=\frac{J^2}{b^2} 
\end{equation*}
\begin{equation*}
	s=1: \indent 1=A(J+b)+B(J+b)+C \indent \Rightarrow 1=A(J+b)+\frac{1}{b}(J+b)+\frac{J^2}{b^2} \indent \Rightarrow A=\frac{-J}{b^2}
\end{equation*}
Plug these values for $A$, $B$, and $C$ into the expanded $\theta(s)$
\begin{eqnarray*}
	\theta (s) & = & \frac{\frac{-J}{b^2}}{s}+\frac{\frac{1}{b}}{s^2}+\frac{\frac{J^2}{b^2}}{Js+b}\\
	&=&\frac{\frac{-J}{b^2}}{s}+\frac{\frac{1}{b}}{s^2}+\frac{\frac{J}{b^2}}{s+\frac{b}{J}}
\end{eqnarray*}
and finally take the inverse Laplace Transform:
\begin{equation*}
	\theta(t)=\frac{-J}{b^2}+\frac{1}{b}t+\frac{J}{b^2}e^{-\frac{b}{J}t}, t \geq 0
\end{equation*}

\noindent Although the first term in the solution $\theta(t)$ is a constant and the third term decays toward zero, the second term, $\frac{1}{b}t$, grows toward infinity as time goes to infinity. Therefore, \textit{using a unit step torque input is not an effective way to make a wind turbine go to a constant angular position.} This result makes sense: as long as there is a constant, nonzero torque applied to this rotational system, we expect it to keep moving; in fact, that's what's happening when the wind is applying a torque to the system for power production.\\

\textbf{\underline{Challenge}}

How does the answer change if the turbine has a non-zero initial velocity \(\dot{\theta}(0^-) = -\pi \) rad/s?\\

\textbf{\underline{Challenge Approach}}

The initial velocity is handled when we first take the Laplace transform, recalling that
\begin{equation*}
	L \bigg\{\frac{d^2}{dt^2}f(t)\bigg \} = s^2F(s)-sf(0^-)-\frac{df}{dt}(0^-)
\end{equation*}
Therefore, we have
\begin{equation*}
	s^2J\theta(s) +\pi + sb\theta(s) = \frac{1}{s}
\end{equation*}
\begin{equation*}
	\theta(s)(Js^2+bs)=\frac{1-\pi s}{s}
\end{equation*}
\begin{equation*}
\theta(s)=\frac{1-\pi s}{s(Js^2+bs)}
\end{equation*}

\noindent And we complete the problem in the same manner as the initial problem - using partial fraction expansion, solving for numerator coefficients, and taking the inverse Laplace Transform with this new numerator.



\section{Lecture Highlights}
The primary takeaways from this article include
\begin{enumerate}
\setlength{\itemsep}{5pt}
\setlength{\parskip}{0pt}
\setlength{\parsep}{0pt}
\item When the Laplace domain function has repeated roots in the denominator, the structure of the partial fraction expansion requires a term for each root. In this case, solving for the coefficients in the partial fraction expansion requires a little more care than in the previous lecture.
\item If there are repeated roots in the Laplace domain, you will find the multiplier $t$ in your time domain solution.
\end{enumerate}

\section{Quiz Yourself}

\subsection{Questions}


\begin{enumerate}
\setlength{\itemsep}{5pt}
\setlength{\parskip}{0pt}
\setlength{\parsep}{0pt}
% \item Find the solution to the differential equation
% \[
% \ddot{x} + 2 \dot{x} +3x = 1 \quad x(0)=1,\dot{x}(0)=0
% \]
% 
% \item Consider the following rotational system. A torque is applied as shown in the figure. If the system is initially at rest at $t=0$ and a torque of $\tau=1$ Nm is applied for $t\geq 0$, what is the time response of the angular position of the spring?\vspace{.1in}
% \begin{center}
% \begin{tikzpicture}[scale=1.75,inner sep=0pt,outer sep=0pt,very thick]

\draw (2.5,0) node (M1) {\begin{tikzpicture}
    \draw[very thick] (.5,0) node[cylinder,draw,shape aspect=.55,minimum width=1cm,minimum height=1.5cm] (J) {$J$};
    \draw[->] (-.2,.5) node[above] {$\theta$}  .. controls  ++(-.15,-.3) and ++(-.15,.3) ..  ++(0,-1);
    \draw[->] (1.4,-.5) node[below] {$\tau$}  .. controls  ++(.15,.3) and ++(.15,-.3) ..  ++(0,1);
    \draw (.5,-1) node {$J\ddot{\theta}=\tau$};
\end{tikzpicture}};
\draw (2.5,0) node {$J$} ++(0,-.6) node {\small arm};
\draw[->] (M1.135) node[above] {$\theta$} ++(-.1,-.1) .. controls  ++(-.15,-.3) and ++(-.15,.3) ..  ++(0,-.8);
\draw[->] (M1.135) ++(-.3,0) node[above=2pt] {$\tau$} ++(-.1,-.1) .. controls  ++(-.15,-.3) and ++(-.15,.3) ..  ++(0,-.8);
\draw (4,.5) node (K) {\begin{tikzpicture}
\draw (.75,0) node[inner sep=0,outer sep=0] (K1) {\begin{tikzpicture}
\draw (.75,0) node[inner sep=0,outer sep=0] (K1) {\input{\mainfolder/DrawingElements/MechanicalElements/spring.tex}};
\draw (K1)  node[above=6pt] {$k$};
\draw[very thick] (K1.180) -- ++(-.2,0);
\draw[very thick] (K1.0) -- ++(0.2,0);
\draw[<-,thick] (K1.0) ++(.2,0) -- ++(.5,0) node[right] {$f$};
\draw[<-,thick] (K1.180) ++(-.2,0) -- ++(-.5,0) node[left] {$f$};
\draw[|->,thick] (K1.180) ++(-.2,.4) node[above=2pt] {$x_{1}$} -- ++(.5,0);  
\draw[|->,thick] (K1.0) ++(.2,.4) node[above=2pt] {$x_{2}$} -- ++(.5,0);  
\draw<2-> (K1) ++(0,-.6) node {$f=k(x_{1}-x_{2})$};
\end{tikzpicture}
};
\draw (K1)  node[above=6pt] {$k$};
\draw[very thick] (K1.180) -- ++(-.2,0);
\draw[very thick] (K1.0) -- ++(0.2,0);
\draw[<-,thick] (K1.0) ++(.2,0) -- ++(.5,0) node[right] {$f$};
\draw[<-,thick] (K1.180) ++(-.2,0) -- ++(-.5,0) node[left] {$f$};
\draw[|->,thick] (K1.180) ++(-.2,.4) node[above=2pt] {$x_{1}$} -- ++(.5,0);  
\draw[|->,thick] (K1.0) ++(.2,.4) node[above=2pt] {$x_{2}$} -- ++(.5,0);  
\draw<2-> (K1) ++(0,-.6) node {$f=k(x_{1}-x_{2})$};
\end{tikzpicture}
};
\draw (4,-.5) node (D) {\begin{tikzpicture}
\draw[very thick] (-.2,0) -- (0,0);
\draw (.75,0) node {\begin{tikzpicture}
\draw[very thick] (-.2,0) -- (0,0);
\draw (.75,0) node {\input{\mainfolder/DrawingElements/MechanicalElements/damper.tex}};
\draw (.75,0) node[above=9pt] {$b$};
\draw[very thick] (1.5,0) -- ++(.2,0);
    \draw[<-,thick] (1.5,0) ++(.2,0) -- ++(.5,0) node[right] {$f$};
    \draw[<-,thick] (-.2,0) -- ++(-.5,0) node[left] {$f$};
    \draw[|->,thick] (-.2,.4) node[above=2pt] {$x_{1}$} -- ++(.5,0);  
    \draw[|->,thick] (1.7,.4) node[above=2pt] {$x_{2}$} -- ++(.5,0);  
    \draw (.6,-.6) node {$x=x_{1}-x_{2}$};
  %  \draw (.6,-1.2) node {$f=b\dot{x}$};
\end{tikzpicture}};
\draw (.75,0) node[above=9pt] {$b$};
\draw[very thick] (1.5,0) -- ++(.2,0);
    \draw[<-,thick] (1.5,0) ++(.2,0) -- ++(.5,0) node[right] {$f$};
    \draw[<-,thick] (-.2,0) -- ++(-.5,0) node[left] {$f$};
    \draw[|->,thick] (-.2,.4) node[above=2pt] {$x_{1}$} -- ++(.5,0);  
    \draw[|->,thick] (1.7,.4) node[above=2pt] {$x_{2}$} -- ++(.5,0);  
    \draw (.6,-.6) node {$x=x_{1}-x_{2}$};
  %  \draw (.6,-1.2) node {$f=b\dot{x}$};
\end{tikzpicture}};
\draw (4,.5) node[above=.25in] {$k$};
\draw (4,-.5) node[above=.25in] {$b$};
\draw (5.5,0) node[rotate=180] (M2) {\input{\mainfolder/DrawingElements/MechanicalElements/ground.tex}};


\draw (M1.0) |- (D.180);
\draw (M1.0) |- (K.180);
\draw (D.0) -- ++(.25,0) |- (M2.0);
\draw (K.0) -- ++(.25,0) |- (M2.0);
\end{tikzpicture}

% \end{center}\vspace{.1in}
\end{enumerate}

\subsection{Solutions}
\begin{enumerate}
\setlength{\itemsep}{5pt}
\setlength{\parskip}{0pt}
\setlength{\parsep}{0pt}
%\item \rule{0pt}{12pt}\\
% \begin{center}
% \includegraphics[width=5in]{quizfigures/1asoln}\\
% \includegraphics[width=5in]{quizfigures/1asoln}
% \end{center}
% \item \rule{0pt}{12pt}\\
% \begin{center}
% \includegraphics[width=4in]{quizfigures/2asoln}\\
% \includegraphics[width=4in]{quizfigures/2bsoln}\\
% \includegraphics[width=4in]{quizfigures/2csoln}\\
% \includegraphics[width=4in]{quizfigures/2dsoln}
% \end{center}
\end{enumerate}

\section{Resources}
Since this lecture is a continuation of Lecture \SolvingDifferentialEquationsINumber,~\SolvingDifferentialEquationsIName, see this lecture for the relevant resources.


\end{document}


