A motor is used to rotate a satellite dish that is supported by shocks. This is modeled as a rotational inertia and spring. Find the transfer function with input motor voltage $v_{a}$ and output rotational position $\theta$. Simplify your answer to a ratio of polynomials.
\begin{center}
\begin{tikzpicture}[scale=1.75,inner sep=0pt,outer sep=0pt,very thick]
\draw (-1,-1) node (a) {}; 

\draw (-1,0) node (Va) { \input{\mainfolder/DrawingElements/CircuitElements/voltagesource.tex}};
\draw (Va.180) node[left=6pt] {$v_{a}$};
\draw (0,1) node (R) {\input{\mainfolder/DrawingElements/CircuitElements/resistor.tex}};
\draw (R.90) node[above=2pt] {$R_{a}$};
\draw (R) node[below=.15in] {$\begin{matrix} \longrightarrow \\ i_{a}\end{matrix}$};
\draw (1,1) node (L) {\input{\mainfolder/DrawingElements/CircuitElements/inductor.tex}};
\draw (L.90) node[above=2pt] {$L_{a}$};
\draw (2,0) node[circle,draw] (Vb) {\rule{0pt}{30pt}};
\draw (3.5,0) node (M1) {\begin{tikzpicture}
    \draw[very thick] (.5,0) node[cylinder,draw,shape aspect=.55,minimum width=1cm,minimum height=1.5cm] (J) {$J$};
    \draw[->] (-.2,.5) node[above] {$\theta$}  .. controls  ++(-.15,-.3) and ++(-.15,.3) ..  ++(0,-1);
    \draw[->] (1.4,-.5) node[below] {$\tau$}  .. controls  ++(.15,.3) and ++(.15,-.3) ..  ++(0,1);
    \draw (.5,-1) node {$J\ddot{\theta}=\tau$};
\end{tikzpicture}};
\draw (3.5,0) node {$J$} ++(0,-.6) node {\small \textsf{dish}};
\draw[->] (M1.135) node[above] {$\theta$} ++(-.1,-.1) .. controls  ++(-.15,-.3) and ++(-.15,.3) ..  ++(0,-.8);
%\draw[->] (M1.135) ++(-.3,0) node[above=2pt] {$\tau$} ++(-.1,-.1) .. controls  ++(-.15,-.3) and ++(-.15,.3) ..  ++(0,-.8);
\draw (5,0) node (K) {\begin{tikzpicture}
\draw (.75,0) node[inner sep=0,outer sep=0] (K1) {\begin{tikzpicture}
\draw (.75,0) node[inner sep=0,outer sep=0] (K1) {\begin{tikzpicture}
\draw (.75,0) node[inner sep=0,outer sep=0] (K1) {\input{\mainfolder/DrawingElements/MechanicalElements/spring.tex}};
\draw (K1)  node[above=6pt] {$k$};
\draw[very thick] (K1.180) -- ++(-.2,0);
\draw[very thick] (K1.0) -- ++(0.2,0);
\draw[<-,thick] (K1.0) ++(.2,0) -- ++(.5,0) node[right] {$f$};
\draw[<-,thick] (K1.180) ++(-.2,0) -- ++(-.5,0) node[left] {$f$};
\draw[|->,thick] (K1.180) ++(-.2,.4) node[above=2pt] {$x_{1}$} -- ++(.5,0);  
\draw[|->,thick] (K1.0) ++(.2,.4) node[above=2pt] {$x_{2}$} -- ++(.5,0);  
\draw<2-> (K1) ++(0,-.6) node {$f=k(x_{1}-x_{2})$};
\end{tikzpicture}
};
\draw (K1)  node[above=6pt] {$k$};
\draw[very thick] (K1.180) -- ++(-.2,0);
\draw[very thick] (K1.0) -- ++(0.2,0);
\draw[<-,thick] (K1.0) ++(.2,0) -- ++(.5,0) node[right] {$f$};
\draw[<-,thick] (K1.180) ++(-.2,0) -- ++(-.5,0) node[left] {$f$};
\draw[|->,thick] (K1.180) ++(-.2,.4) node[above=2pt] {$x_{1}$} -- ++(.5,0);  
\draw[|->,thick] (K1.0) ++(.2,.4) node[above=2pt] {$x_{2}$} -- ++(.5,0);  
\draw<2-> (K1) ++(0,-.6) node {$f=k(x_{1}-x_{2})$};
\end{tikzpicture}
};
\draw (K1)  node[above=6pt] {$k$};
\draw[very thick] (K1.180) -- ++(-.2,0);
\draw[very thick] (K1.0) -- ++(0.2,0);
\draw[<-,thick] (K1.0) ++(.2,0) -- ++(.5,0) node[right] {$f$};
\draw[<-,thick] (K1.180) ++(-.2,0) -- ++(-.5,0) node[left] {$f$};
\draw[|->,thick] (K1.180) ++(-.2,.4) node[above=2pt] {$x_{1}$} -- ++(.5,0);  
\draw[|->,thick] (K1.0) ++(.2,.4) node[above=2pt] {$x_{2}$} -- ++(.5,0);  
\draw<2-> (K1) ++(0,-.6) node {$f=k(x_{1}-x_{2})$};
\end{tikzpicture}
};
\draw (K.90) node[above=2pt] {$k$};
\draw (6,0) node[rotate=180] (gnd) {\input{\mainfolder/DrawingElements/MechanicalElements/ground.tex}};


\draw (M1.0) -- (K.180);
\draw (K.0) -- (gnd.0);
\draw (2,0) -- (M1.180);
\draw (R.0) -- (L.180);
\draw (L.0) -| (Vb.90);
\draw (Va.90) |- (R.180);
\draw (Va) -- (a);
\draw (a.0) -| (Vb.-90);
\end{tikzpicture}
\end{center}

