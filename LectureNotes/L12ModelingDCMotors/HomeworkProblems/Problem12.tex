A ceiling fan usually uses a single phase AC induction motor, but for this problem let's suppose that a brushed DC motor is used instead. The DC motor has motor constants $K_{t}=K_{e}=3$, the internal resistance is $1$ Ohm, and the internal inductance can be neglected. The inertia of a fan blades and housing is 2 kg m$^2$, while the air resistance creates a torque that is proportional to the angular velocity, and can be modeled using a rotational damper with damping coefficient of 1 N m s rad$^{-1}$
\begin{center}
\begin{minipage}{3in}
\includegraphics[width=3in]{\mainfolder/LectureNotes/\lecturefolder/HomeworkProblems/Problem12/Ceiling_fan_with_light.png}
\end{minipage}
\begin{minipage}{3in}
\resizebox{3in}{!}{
\begin{tikzpicture}[scale=1.3,inner sep=0pt,outer sep=0pt,very thick]
\draw (0,0) node[fill=black] (a) {}; 
\draw (0,-2) node[fill=black] (d) {};
\draw (1.5,-2) node[fill=black] (f) {};
 
\draw (2,0) node (R1) {\input{\mainfolder/DrawingElements/CircuitElements/resistor.tex}};
\draw (R1) node[below=.2in] {$R_{a}$};
%\draw (2.5,0) node (L1) {\input{../../../Tikzmodeling/CircuitElements/inductor.tex}};
%\draw (2.5,0) node[above=.2in] {$L_{a}$};
\draw (3,-2) node[rotate=-90] (Rot) {\input{\mainfolder/DrawingElements/CircuitElements/rotor.tex}}; 
\draw (Rot) ++(0,-1.7) node[rotate=-90] (I) {\input{\mainfolder/DrawingElements/MechanicalElements/inertia2.tex}};
\draw (I.180) -- (Rot.0);
\draw (I) node {$J$};
\draw[->] (I.180) ++(.5,-.05)  node[right=2pt] {$\theta$}  .. controls  ++(-.3,.15) and ++(.3,.15) ..  ++(-1,0);
\draw[->] (I.180) ++(.5,.15) node[right=2pt] {$\tau_{m}$}  .. controls  ++(-.3,.15) and ++(.3,.15) ..  ++(-1,0);
\draw (Rot) ++(0,-3) node[rotate=-90] (D) {\begin{tikzpicture}
\draw[very thick] (-.2,0) -- (0,0);
\draw (.75,0) node {\begin{tikzpicture}
\draw[very thick] (-.2,0) -- (0,0);
\draw (.75,0) node {\begin{tikzpicture}
\draw[very thick] (-.2,0) -- (0,0);
\draw (.75,0) node {\input{\mainfolder/DrawingElements/MechanicalElements/damper.tex}};
\draw (.75,0) node[above=9pt] {$b$};
\draw[very thick] (1.5,0) -- ++(.2,0);
    \draw[<-,thick] (1.5,0) ++(.2,0) -- ++(.5,0) node[right] {$f$};
    \draw[<-,thick] (-.2,0) -- ++(-.5,0) node[left] {$f$};
    \draw[|->,thick] (-.2,.4) node[above=2pt] {$x_{1}$} -- ++(.5,0);  
    \draw[|->,thick] (1.7,.4) node[above=2pt] {$x_{2}$} -- ++(.5,0);  
    \draw (.6,-.6) node {$x=x_{1}-x_{2}$};
  %  \draw (.6,-1.2) node {$f=b\dot{x}$};
\end{tikzpicture}};
\draw (.75,0) node[above=9pt] {$b$};
\draw[very thick] (1.5,0) -- ++(.2,0);
    \draw[<-,thick] (1.5,0) ++(.2,0) -- ++(.5,0) node[right] {$f$};
    \draw[<-,thick] (-.2,0) -- ++(-.5,0) node[left] {$f$};
    \draw[|->,thick] (-.2,.4) node[above=2pt] {$x_{1}$} -- ++(.5,0);  
    \draw[|->,thick] (1.7,.4) node[above=2pt] {$x_{2}$} -- ++(.5,0);  
    \draw (.6,-.6) node {$x=x_{1}-x_{2}$};
  %  \draw (.6,-1.2) node {$f=b\dot{x}$};
\end{tikzpicture}};
\draw (.75,0) node[above=9pt] {$b$};
\draw[very thick] (1.5,0) -- ++(.2,0);
    \draw[<-,thick] (1.5,0) ++(.2,0) -- ++(.5,0) node[right] {$f$};
    \draw[<-,thick] (-.2,0) -- ++(-.5,0) node[left] {$f$};
    \draw[|->,thick] (-.2,.4) node[above=2pt] {$x_{1}$} -- ++(.5,0);  
    \draw[|->,thick] (1.7,.4) node[above=2pt] {$x_{2}$} -- ++(.5,0);  
    \draw (.6,-.6) node {$x=x_{1}-x_{2}$};
  %  \draw (.6,-1.2) node {$f=b\dot{x}$};
\end{tikzpicture}};
\draw (D) node[right=12pt] {$b$};
\draw  (Rot) ++(0,-4) node[rotate=90] (gnd) {\input{\mainfolder/DrawingElements/MechanicalElements/ground.tex}};

%\draw[->] (3.35,-.1) -- node[left=1pt] {$i_{a}$} ++(0,-.5); 
\draw (Rot) node[above=.2in] {$\begin{matrix} - & v_{b} & +\end{matrix}$};
\draw (0,-1) node (V) {\input{\mainfolder/DrawingElements/CircuitElements/voltagesource.tex}};
\draw (0,-1) node[left=.25in]{$v_{a}$};

%\draw[->] (1.5,-.5) -- node[pos=.5,below=4pt] {$i$} ++(1,0); 
\draw (I.0) -- (D.180);
\draw (D.0) -- (gnd.0);
\draw (V) |- (R1);
%\draw (R1) -- (L1);
%\draw (L1) -| (Rot.90);
\draw (R1) -| (Rot.90);
\draw (Rot.-90) |- (f);
\draw (f) -- (d);
\draw (a) -- (V);
\draw (d) -- (V);
\end{tikzpicture}
}
\end{minipage}
\end{center}
Find the transfer function $\frac{\theta(s)}{V_{a}(s)}$.

