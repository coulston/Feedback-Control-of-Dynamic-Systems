Using the provided log scales, sketch the Bode plots using linear approximation rules. For full credit, label and scale each y-axis. Indicate the slope of each segment, and the magnitude and phase at each break point.
\vspace{.2in}

\noindent\begin{minipage}{3.25in}
(a) Plot the Bode plot for $G_1(s) = \frac{3}{s}$
\begin{center}
\resizebox{3in}{!}{
\begin{tikzpicture}
\draw (0,0) node {\includegraphics[width=3.25in]{\mainfolder/LectureNotes/\lecturefolder/HomeworkProblems/blankbode}};
\draw (-1.95,.3) node {0.1};
\draw (0,.3) node {1};
\draw (1.9,.3) node {10};
\draw (3.8,.3) node {100};
\draw (4.4,-.1) node {$\omega$ (rad/s)};
\draw (-1.95,-3.7) node {0.1};
\draw (0,-3.7) node {1};
\draw (1.9,-3.7) node {10};
\draw (3.8,-3.7) node {100};
\draw (4.4,-4.1) node {$\omega$ (rad/s)};
\end{tikzpicture}
}

\end{center}
\end{minipage}
\begin{minipage}{3.25in}
(b) Plot the Bode plot for $G_2(s) = \frac{1}{s+3}$
\begin{center}
\resizebox{3in}{!}{
\begin{tikzpicture}
\draw (0,0) node {\includegraphics[width=3.25in]{\mainfolder/LectureNotes/\lecturefolder/HomeworkProblems/blankbode}};
\draw (-1.95,.3) node {0.1};
\draw (0,.3) node {1};
\draw (1.9,.3) node {10};
\draw (3.8,.3) node {100};
\draw (4.4,-.1) node {$\omega$ (rad/s)};
\draw (-1.95,-3.7) node {0.1};
\draw (0,-3.7) node {1};
\draw (1.9,-3.7) node {10};
\draw (3.8,-3.7) node {100};
\draw (4.4,-4.1) node {$\omega$ (rad/s)};
\end{tikzpicture}
}
\end{center}

\end{minipage}\vspace{.2in}\\
\noindent(c) If the input to system $G_{2}$ is $\sin(10t)u(t)$, use your Bode Plot sketch to estimate the steady state output.

