
\newboolean{PrintExample}
\setboolean{PrintExample}{false}
\newcommand{\tryit}[2]{\ifthenelse{\boolean{PrintExample}}{{{\color{gray} #2}}}{#1}}
\definecolor{lightgray}{rgb}{.85,.85,.85}

%
% Course Information
% 
\newcommand{\instituteshort}{CSM}
\newcommand{\institutelong}{Department of Electrical Engineering and Computer Science\\Colorado School of Mines}
\newcommand{\instructorshort}{}
\newcommand{\instructorlong}{Ayoade Adewole, Christopher Coulston, Kathryn Johnson, Alok Sarwal, and  Tyrone Vincent}
\newcommand{\semester}{Fall}
\newcommand{\courseyear}{2016}
\newcommand{\semesteryear}{\semester~\courseyear}
\newcommand{\semesteryearlink}{\semester\courseyear}
\newcommand{\course}{EENG307}
\newcommand{\coursename}{Intro to Feedback Control}
\newcommand{\license}{\footnote{\tiny This work is licensed under the Creative Commons Attribution-NonCommercial-ShareAlike 3.0 Unported License. To view a copy of this license, visit http://creativecommons.org/licenses/by-nc-sa/3.0/ or send a letter to Creative Commons, 171 Second Street, Suite 300, San Francisco, California, 94105, USA.}}
\newcommand{\credits}{\footnote{\tiny Developed and edited by Tyrone Vincent and Kathryn Johnson, Colorado School of Mines, with contributions from Kevin Moore, CSM and Matt Kupilik, University of Alaska, Anchorage }}

%
% Lecture Names and Numbers
%
\newcommand{\ModelingMechanicalSystemsNumber}{1}
\newcommand{\ModelingMechanicalSystemsName}{Modeling Mechanical Systems}
\newcommand{\ModelingMechanicalSystemsShortName}{Mechanical Systems}
\newcommand{\ElectricalSystemsNumber}{2}
\newcommand{\ElectricalSystemsName}{Modeling Electrical Systems}
\newcommand{\ElectricalSystemsShortName}{Electrical Systems}
\newcommand{\LaplaceTransformReviewNumber}{3}
\newcommand{\LaplaceTransformReviewName}{Laplace Transform Review}
\newcommand{\LaplaceTransformReviewShortName}{Laplace Transform Review}
\newcommand{\SolvingDifferentialEquationsINumber}{4}
\newcommand{\SolvingDifferentialEquationsIName}{Solving Differential Equations using Laplace Transforms, Part I}
\newcommand{\SolvingDifferentialEquationsIShortName}{Solving Differential Equations I}
\newcommand{\SolvingDifferentialEquationsIINumber}{5}
\newcommand{\SolvingDifferentialEquationsIIName}{Solving Differential Equations using Laplace Transforms, Part II}
\newcommand{\SolvingDifferentialEquationsIIShortName}{Solving Differential Equations II}
\newcommand{\ImpedanceNumber}{6}
\newcommand{\ImpedanceName}{Impedance and Transfer Functions}
\newcommand{\ImpedanceShortName}{Impedance}
\newcommand{\FluidandAnalogousSystemsNumber}{7}
\newcommand{\FluidandAnalogousSystemsName}{Fluid Systems and System Analogies} 
\newcommand{\FluidandAnalogousSystemsShortName}{Fluid Systems and Analogies}
\newcommand{\MechanicalImpedanceNumber}{8}
\newcommand{\MechanicalImpedanceName}{Mechanical Impedance}
\newcommand{\MechanicalImpedanceShortName}{Mechanical Impedance}
\newcommand{\ThermalSystemsNumber}{9}
\newcommand{\ThermalSystemsName}{Thermal Systems}
\newcommand{\ThermalSystemsShortName}{Thermal Systems}
\newcommand{\BlockDiagramsNumber}{10}
\newcommand{\BlockDiagramsName}{Block Diagrams}
\newcommand{\BlockDiagramsShortName}{Block Diagrams}
\newcommand{\MotorsandHydraulicActuatorsNumber}{11}
\newcommand{\MotorsandHydraulicActuatorsName}{Motors and Hydraulic Actuators}
\newcommand{\MotorsandHydraulicActuatorsShortName}{Motors and Hydraulics}
\newcommand{\FirstOrderResponseNumber}{12}
\newcommand{\FirstOrderResponseName}{Time Response of First Order Systems}
\newcommand{\FirstOrderResponseShortName}{First Order Systems}
\newcommand{\SecondOrderResponseNumber}{13}
\newcommand{\SecondOrderResponseName}{Time Response of Second Order Systems}
\newcommand{\SecondOrderResponseShortName}{Second Order Systems}
\newcommand{\HigherOrderResponseNumber}{14}
\newcommand{\HigherOrderResponseName}{Time Response of Higher Order Systems}
\newcommand{\HigherOrderResponseShortName}{Higher Order Systems}
\newcommand{\SystemIdentificationNumber}{15}
\newcommand{\SystemIdentificationName}{System Identification}
\newcommand{\SystemIdentificationShortName}{System Identification}
\newcommand{\StabilityandRouthHurwitzNumber}{16}
\newcommand{\StabilityandRouthHurwitzName}{Stability and Routh Hurwitz Criterion}
\newcommand{\StabilityandRouthHurwitzShortName}{Stability}
\newcommand{\DisturbancesandSteadyStateErrorNumber}{17}
\newcommand{\DisturbancesandSteadyStateErrorName}{Disturbances and Steady State Error}
\newcommand{\DisturbancesandSteadyStateErrorShortName}{Steady State Error}
\newcommand{\ReferenceSteadyStateErrorNumber}{18}
\newcommand{\ReferenceSteadyStateErrorName}{Reference Steady State Error and System Type}
\newcommand{\ReferenceSteadyStateErrorShortName}{Reference SSE}
\newcommand{\PIDControlNumber}{19}
\newcommand{\PIDControlName}{Proportional, Integral, and Derivative (PID) Control}
\newcommand{\PIDControlShortName}{PID Control}
\newcommand{\RootLocusINumber}{20}
\newcommand{\RootLocusIName}{Introduction to Root Locus}
\newcommand{\RootLocusIShortName}{Root Locus I}
\newcommand{\RootLocusIINumber}{21}
\newcommand{\RootLocusIIName}{Root Locus Examples}
\newcommand{\RootLocusIIShortName}{Root Locus II}
\newcommand{\RootLocusIIINumber}{22}
\newcommand{\RootLocusIIIName}{PD Design Using Root Locus}
\newcommand{\RootLocusIIIShortName}{Root Locus III}

%
% Matlab compatible quote symbol
%
\newcommand{\T}{\textquotesingle\ignorespaces}

% Notation
\newcommand{\steparg}[1]{u(#1)}
\providecommand{\define}[3]{\newcommand{#1}{{#2}}}

\define{\control}{C}{Controller, represented as transfer function}
\define{\controlimp}{c}{Controller, represented as impuse response} 
\define{\inpt}{r}{Variable to use when representing a generic input}
\define{\inptLT}{R}{Variable to use when representing the Laplace Transform of the input}
\define{\outpt}{y}{Variable to use when representing a generic output}
\define{\outptLT}{Y}{Variable to use when representing the Laplace Transform of the output}
\define{\plant}{G}{System to be controlled, represented as transfer function}
\define{\plantimp}{g}{System to be controlled, represented as impulse response}
\define{\state}{x}{State variable, in time domain}
\define{\stateLT}{X}{State variables in Laplace domain}
\define{\s}{s}{Laplace variable}
\define{\q}{q}{Unit delay operator}
\define{\step}{u}{Variable for Unit Step Signal}
\define{\tr}{t_{r}}{Rise time}
\define{\treqone}{\frac{2.2}{\sigma}}{Equation for rise time, 1st order system}
\define{\treqtwo}{\frac{2.2}{\omega_{n}}}{Equation for rise time, 2nd order system}
\define{\ts}{t_{s}}{Settling Time}
\define{\tseqone}{\frac{4.6}{\sigma}}{Equation for settling time, 1st order system}
\define{\tseqtwo}{\frac{4.6}{\zeta\omega_{n}}}{Equation for settling time, 2nd order system}
\define{\OS}{\%OS}{Percent Overshoot}
\define{\OSeq}{e^{-\zeta\pi/\sqrt{1-\zeta^2}}\times 100\%}{Equation for overshoot}
\define{\PM}{\phi_{PM}}{Phase Margin}
\define{\GM}{k_{GM}}{Gain Margin (default units)}
\define{\GMdB}{GM}{Gain Margin (decibels)}


%
% titleformat
%
\author[\instructorshort]% (optional, use only with lots of authors)
{\instructorlong\credits}
%{T. Vincent\inst{1} \and S.~Another\inst{2}}
% - Use the \inst{?} command only if the authors have different
%   affiliation.

\institute[\instituteshort] % (optional, but mostly needed)
{\institutelong}

\date{\semesteryear} % 

%
% Slide formats
%

% Delete this, if you do not want the table of contents to pop up at
% the beginning of each subsection:
%\AtBeginSection[]
%{
%  \begin{frame}<beamer>{Outline}
%    \tableofcontents[currentsection,currentsubsection]
%  \end{frame}
%}


% If you wish to uncover everything in a step-wise fashion, uncomment
% the following command:

%\beamerdefaultoverlayspecification{<+->}


