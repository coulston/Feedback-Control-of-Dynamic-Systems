
A community water tank with vertical walls and area $A = 100$ m$^2$ is filled by a source $q_{in}(t)$ and drains to supply water to the community through a leaky valve with resistance $R$. 

\begin{enumerate}[(a)]
\setlength{\itemsep}{3pt}
\setlength{\parskip}{0pt}
\setlength{\parsep}{0pt}
\item Sketch an ideal element model for this system using tank and valve components and labeling all flows, pressures, height, and any other variables you think you may need to model the system. %\vspace{1.2in}
\item Assume the differential equation modeling the height $h(t)$ of water in the tank is given by
\[
A \frac{dh(t)}{dt}+\frac{\rho g}{R} h(t) = q_{in}(t)
\]
(may or may not be consistent with part (a)), where $R = 9.8$ kg m$^{-4}$ s$^{-1}$ when the valve is closed and leaking, $\rho = 1$ kg/m$^3$ is the density of water, $g = 9.8$ m/s$^2$ is the gravitational constant, and $h_0 = 3$~m is the initial height of water in the tank. If the electrical power goes out and the supply pump stops ($q_{in} = 0$), find $h(t)$. %\vspace{2.8in}
\item How long will it take for the community to have only 10\% of its initial volume of water, or $h = 0.3$~m, remaining? Note the natural log property ln$(e^x) = x$ and leave your answer as a function of `ln'.
\end{enumerate}

