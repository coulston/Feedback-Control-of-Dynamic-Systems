Find the transfer function from input torque $\tau$ to output $\theta$ for the following rotational mechanical system.
\begin{center}
\begin{tikzpicture}[scale=1.75,inner sep=0pt,outer sep=0pt,very thick]
\draw (0,0) node (a) {}; 
\draw (2,0) node (b) {};
\draw (3,0) node (c) {};
\draw (0,-2) node (d) {};
\draw (2,-2) node (e) {};
\draw (3,-2) node (f) {};
 
\draw (0,0) node (gnd1) {\input{\mainfolder/DrawingElements/MechanicalElements/ground.tex}};
\draw (1,0) node (K1) {\begin{tikzpicture}
\draw (.75,0) node[inner sep=0,outer sep=0] (K1) {\begin{tikzpicture}
\draw (.75,0) node[inner sep=0,outer sep=0] (K1) {\begin{tikzpicture}
\draw (.75,0) node[inner sep=0,outer sep=0] (K1) {\input{\mainfolder/DrawingElements/MechanicalElements/spring.tex}};
\draw (K1)  node[above=6pt] {$k$};
\draw[very thick] (K1.180) -- ++(-.2,0);
\draw[very thick] (K1.0) -- ++(0.2,0);
\draw[<-,thick] (K1.0) ++(.2,0) -- ++(.5,0) node[right] {$f$};
\draw[<-,thick] (K1.180) ++(-.2,0) -- ++(-.5,0) node[left] {$f$};
\draw[|->,thick] (K1.180) ++(-.2,.4) node[above=2pt] {$x_{1}$} -- ++(.5,0);  
\draw[|->,thick] (K1.0) ++(.2,.4) node[above=2pt] {$x_{2}$} -- ++(.5,0);  
\draw<2-> (K1) ++(0,-.6) node {$f=k(x_{1}-x_{2})$};
\end{tikzpicture}
};
\draw (K1)  node[above=6pt] {$k$};
\draw[very thick] (K1.180) -- ++(-.2,0);
\draw[very thick] (K1.0) -- ++(0.2,0);
\draw[<-,thick] (K1.0) ++(.2,0) -- ++(.5,0) node[right] {$f$};
\draw[<-,thick] (K1.180) ++(-.2,0) -- ++(-.5,0) node[left] {$f$};
\draw[|->,thick] (K1.180) ++(-.2,.4) node[above=2pt] {$x_{1}$} -- ++(.5,0);  
\draw[|->,thick] (K1.0) ++(.2,.4) node[above=2pt] {$x_{2}$} -- ++(.5,0);  
\draw<2-> (K1) ++(0,-.6) node {$f=k(x_{1}-x_{2})$};
\end{tikzpicture}
};
\draw (K1)  node[above=6pt] {$k$};
\draw[very thick] (K1.180) -- ++(-.2,0);
\draw[very thick] (K1.0) -- ++(0.2,0);
\draw[<-,thick] (K1.0) ++(.2,0) -- ++(.5,0) node[right] {$f$};
\draw[<-,thick] (K1.180) ++(-.2,0) -- ++(-.5,0) node[left] {$f$};
\draw[|->,thick] (K1.180) ++(-.2,.4) node[above=2pt] {$x_{1}$} -- ++(.5,0);  
\draw[|->,thick] (K1.0) ++(.2,.4) node[above=2pt] {$x_{2}$} -- ++(.5,0);  
\draw<2-> (K1) ++(0,-.6) node {$f=k(x_{1}-x_{2})$};
\end{tikzpicture}
};
\draw (1,0) node[above=.2in] {$k$};
\draw (2.5,0) node (M1) {\begin{tikzpicture}
    \draw[very thick] (.5,0) node[cylinder,draw,shape aspect=.55,minimum width=1cm,minimum height=1.5cm] (J) {$J$};
    \draw[->] (-.2,.5) node[above] {$\theta$}  .. controls  ++(-.15,-.3) and ++(-.15,.3) ..  ++(0,-1);
    \draw[->] (1.4,-.5) node[below] {$\tau$}  .. controls  ++(.15,.3) and ++(.15,-.3) ..  ++(0,1);
    \draw (.5,-1) node {$J\ddot{\theta}=\tau$};
\end{tikzpicture}};
\draw (2.5,0) node {$J$};
\draw[->] (M1.135) node[above] {$\theta$} ++(-.1,-.1) .. controls  ++(-.15,-.3) and ++(-.15,.3) ..  ++(0,-.8);
\draw (4,0) node (D) {\begin{tikzpicture}
\draw[very thick] (-.2,0) -- (0,0);
\draw (.75,0) node {\begin{tikzpicture}
\draw[very thick] (-.2,0) -- (0,0);
\draw (.75,0) node {\begin{tikzpicture}
\draw[very thick] (-.2,0) -- (0,0);
\draw (.75,0) node {\input{\mainfolder/DrawingElements/MechanicalElements/damper.tex}};
\draw (.75,0) node[above=9pt] {$b$};
\draw[very thick] (1.5,0) -- ++(.2,0);
    \draw[<-,thick] (1.5,0) ++(.2,0) -- ++(.5,0) node[right] {$f$};
    \draw[<-,thick] (-.2,0) -- ++(-.5,0) node[left] {$f$};
    \draw[|->,thick] (-.2,.4) node[above=2pt] {$x_{1}$} -- ++(.5,0);  
    \draw[|->,thick] (1.7,.4) node[above=2pt] {$x_{2}$} -- ++(.5,0);  
    \draw (.6,-.6) node {$x=x_{1}-x_{2}$};
  %  \draw (.6,-1.2) node {$f=b\dot{x}$};
\end{tikzpicture}};
\draw (.75,0) node[above=9pt] {$b$};
\draw[very thick] (1.5,0) -- ++(.2,0);
    \draw[<-,thick] (1.5,0) ++(.2,0) -- ++(.5,0) node[right] {$f$};
    \draw[<-,thick] (-.2,0) -- ++(-.5,0) node[left] {$f$};
    \draw[|->,thick] (-.2,.4) node[above=2pt] {$x_{1}$} -- ++(.5,0);  
    \draw[|->,thick] (1.7,.4) node[above=2pt] {$x_{2}$} -- ++(.5,0);  
    \draw (.6,-.6) node {$x=x_{1}-x_{2}$};
  %  \draw (.6,-1.2) node {$f=b\dot{x}$};
\end{tikzpicture}};
\draw (.75,0) node[above=9pt] {$b$};
\draw[very thick] (1.5,0) -- ++(.2,0);
    \draw[<-,thick] (1.5,0) ++(.2,0) -- ++(.5,0) node[right] {$f$};
    \draw[<-,thick] (-.2,0) -- ++(-.5,0) node[left] {$f$};
    \draw[|->,thick] (-.2,.4) node[above=2pt] {$x_{1}$} -- ++(.5,0);  
    \draw[|->,thick] (1.7,.4) node[above=2pt] {$x_{2}$} -- ++(.5,0);  
    \draw (.6,-.6) node {$x=x_{1}-x_{2}$};
  %  \draw (.6,-1.2) node {$f=b\dot{x}$};
\end{tikzpicture}};
\draw (4,0) node[above=.25in] {$b$};


\draw (gnd1) -- (K1);
\draw (K1) -- (M1);
\draw (M1.0) -- (D.180);
\draw (D.0) -- ++(.5,0); 
\draw[->] (D.0) ++(.5,.5) node[above] {$\tau$} ++(-.1,-.1) .. controls  ++(-.15,-.3) and ++(-.15,.3) ..  ++(0,-.8);
\end{tikzpicture}
\end{center}
