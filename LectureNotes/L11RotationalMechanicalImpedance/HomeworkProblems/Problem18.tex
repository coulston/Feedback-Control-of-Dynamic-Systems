Draw the impedance network that represents the following fluid system. There is a source of fluid supplied at pressure $p_{in}$. The pressure at the top of the tank is maintained at pressure $p_{T}$, while fluid is discharged to atmospheric pressure $p_{a}$. The gravitational constant is $g$ and the fluid density is $\rho$.
\begin{center}
\begin{tikzpicture}
\draw (.75,0) node[above] (tank1) {\input{\mainfolder/DrawingElements/FluidElements/tank.tex}};
\draw[decorate,decoration={coil,aspect=0,segment length=5.85pt}] (tank1.90) ++(-1.2,-1) -- ++(2.38,0);

\draw (tank1.0) ++(.3,-.64) node (pipe2) {\input{\mainfolder/DrawingElements/FluidElements/valve.tex}};
\draw (tank1.-180) ++(-.3,-.64) node (pipe1) {\input{\mainfolder/DrawingElements/FluidElements/valve.tex}};
\draw (tank1.-90) node{Tank Area: $A_{1}$};
\draw (tank1.-90) ++(0,.75) node[above] {$p_{1}$};
\draw (tank1.90) ++(0,-.75) node[above] {$p_{T}$};
\draw[thick] (tank1.90) ++(-1.5,-.15) -- ++(3,0);


\draw (pipe1.-180) node[left=-.1in] (pipe0_1) {\input{\mainfolder/DrawingElements/FluidElements/pipe.tex}};
\draw (pipe1.90) node[above] {$R_{1}$};
\draw (pipe2.90) node[above] {$R_{2}$};
\draw[->] (pipe2.0) ++(-.5,0) --  ++(.5,0) node[right] {$q_{out}$};
\draw(pipe2.0) ++(1,0) node[right] {$p_{a}$};
\draw (pipe1.180) ++(-.5,0) node[left] {$p_{in}$};
\end{tikzpicture}
\end{center}
