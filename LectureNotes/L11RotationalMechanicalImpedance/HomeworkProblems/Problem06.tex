The Chandra X-ray Observatory is a space observatory sensitive to x-ray sources. To make observations the observatory must be precisely oriented. 
\begin{center}
\includegraphics[width=3.2in]{\mainfolder/LectureNotes/\lecturefolder/HomeworkProblems/problem06/704250main_chandra-telescope_full}
{\tiny source:NASA}
\end{center}
The observatory has a main body connected to solar panels on a lightweight frame. An ideal element model for motion of the the observatory around its major axis is shown in the following figure, where $\tau$ is the rotational torque applied by thrusters or reaction wheels, $\theta$ is the orientation of the main body, $J_1$ is the rotational inertia of the main body, and $J_2$ is the rotational inertia of the solar array, which has a flexible connection to the main body consistent with a damping ratio of $b$ and a spring constant $k$.
\begin{center}
\begin{tikzpicture}[scale=1.75,inner sep=0pt,outer sep=0pt,very thick]
 
\draw (-.5,0) node (M0) {\begin{tikzpicture}
    \draw[very thick] (.5,0) node[cylinder,draw,shape aspect=.55,minimum width=1cm,minimum height=1.5cm] (J) {$J$};
    \draw[->] (-.2,.5) node[above] {$\theta$}  .. controls  ++(-.15,-.3) and ++(-.15,.3) ..  ++(0,-1);
    \draw[->] (1.4,-.5) node[below] {$\tau$}  .. controls  ++(.15,.3) and ++(.15,-.3) ..  ++(0,1);
    \draw (.5,-1) node {$J\ddot{\theta}=\tau$};
\end{tikzpicture}};
\draw (1,.5) node (K1) {\begin{tikzpicture}
\draw (.75,0) node[inner sep=0,outer sep=0] (K1) {\begin{tikzpicture}
\draw (.75,0) node[inner sep=0,outer sep=0] (K1) {\begin{tikzpicture}
\draw (.75,0) node[inner sep=0,outer sep=0] (K1) {\input{\mainfolder/DrawingElements/MechanicalElements/spring.tex}};
\draw (K1)  node[above=6pt] {$k$};
\draw[very thick] (K1.180) -- ++(-.2,0);
\draw[very thick] (K1.0) -- ++(0.2,0);
\draw[<-,thick] (K1.0) ++(.2,0) -- ++(.5,0) node[right] {$f$};
\draw[<-,thick] (K1.180) ++(-.2,0) -- ++(-.5,0) node[left] {$f$};
\draw[|->,thick] (K1.180) ++(-.2,.4) node[above=2pt] {$x_{1}$} -- ++(.5,0);  
\draw[|->,thick] (K1.0) ++(.2,.4) node[above=2pt] {$x_{2}$} -- ++(.5,0);  
\draw<2-> (K1) ++(0,-.6) node {$f=k(x_{1}-x_{2})$};
\end{tikzpicture}
};
\draw (K1)  node[above=6pt] {$k$};
\draw[very thick] (K1.180) -- ++(-.2,0);
\draw[very thick] (K1.0) -- ++(0.2,0);
\draw[<-,thick] (K1.0) ++(.2,0) -- ++(.5,0) node[right] {$f$};
\draw[<-,thick] (K1.180) ++(-.2,0) -- ++(-.5,0) node[left] {$f$};
\draw[|->,thick] (K1.180) ++(-.2,.4) node[above=2pt] {$x_{1}$} -- ++(.5,0);  
\draw[|->,thick] (K1.0) ++(.2,.4) node[above=2pt] {$x_{2}$} -- ++(.5,0);  
\draw<2-> (K1) ++(0,-.6) node {$f=k(x_{1}-x_{2})$};
\end{tikzpicture}
};
\draw (K1)  node[above=6pt] {$k$};
\draw[very thick] (K1.180) -- ++(-.2,0);
\draw[very thick] (K1.0) -- ++(0.2,0);
\draw[<-,thick] (K1.0) ++(.2,0) -- ++(.5,0) node[right] {$f$};
\draw[<-,thick] (K1.180) ++(-.2,0) -- ++(-.5,0) node[left] {$f$};
\draw[|->,thick] (K1.180) ++(-.2,.4) node[above=2pt] {$x_{1}$} -- ++(.5,0);  
\draw[|->,thick] (K1.0) ++(.2,.4) node[above=2pt] {$x_{2}$} -- ++(.5,0);  
\draw<2-> (K1) ++(0,-.6) node {$f=k(x_{1}-x_{2})$};
\end{tikzpicture}
};
\draw (1,-.5) node (D1) {\begin{tikzpicture}
\draw[very thick] (-.2,0) -- (0,0);
\draw (.75,0) node {\begin{tikzpicture}
\draw[very thick] (-.2,0) -- (0,0);
\draw (.75,0) node {\begin{tikzpicture}
\draw[very thick] (-.2,0) -- (0,0);
\draw (.75,0) node {\input{\mainfolder/DrawingElements/MechanicalElements/damper.tex}};
\draw (.75,0) node[above=9pt] {$b$};
\draw[very thick] (1.5,0) -- ++(.2,0);
    \draw[<-,thick] (1.5,0) ++(.2,0) -- ++(.5,0) node[right] {$f$};
    \draw[<-,thick] (-.2,0) -- ++(-.5,0) node[left] {$f$};
    \draw[|->,thick] (-.2,.4) node[above=2pt] {$x_{1}$} -- ++(.5,0);  
    \draw[|->,thick] (1.7,.4) node[above=2pt] {$x_{2}$} -- ++(.5,0);  
    \draw (.6,-.6) node {$x=x_{1}-x_{2}$};
  %  \draw (.6,-1.2) node {$f=b\dot{x}$};
\end{tikzpicture}};
\draw (.75,0) node[above=9pt] {$b$};
\draw[very thick] (1.5,0) -- ++(.2,0);
    \draw[<-,thick] (1.5,0) ++(.2,0) -- ++(.5,0) node[right] {$f$};
    \draw[<-,thick] (-.2,0) -- ++(-.5,0) node[left] {$f$};
    \draw[|->,thick] (-.2,.4) node[above=2pt] {$x_{1}$} -- ++(.5,0);  
    \draw[|->,thick] (1.7,.4) node[above=2pt] {$x_{2}$} -- ++(.5,0);  
    \draw (.6,-.6) node {$x=x_{1}-x_{2}$};
  %  \draw (.6,-1.2) node {$f=b\dot{x}$};
\end{tikzpicture}};
\draw (.75,0) node[above=9pt] {$b$};
\draw[very thick] (1.5,0) -- ++(.2,0);
    \draw[<-,thick] (1.5,0) ++(.2,0) -- ++(.5,0) node[right] {$f$};
    \draw[<-,thick] (-.2,0) -- ++(-.5,0) node[left] {$f$};
    \draw[|->,thick] (-.2,.4) node[above=2pt] {$x_{1}$} -- ++(.5,0);  
    \draw[|->,thick] (1.7,.4) node[above=2pt] {$x_{2}$} -- ++(.5,0);  
    \draw (.6,-.6) node {$x=x_{1}-x_{2}$};
  %  \draw (.6,-1.2) node {$f=b\dot{x}$};
\end{tikzpicture}};
\draw (K1) node[above=.2in] {$k$};
\draw (D1) node[above=.2in] {$b$};
\draw (2.5,0) node (M1) {\begin{tikzpicture}
    \draw[very thick] (.5,0) node[cylinder,draw,shape aspect=.55,minimum width=1cm,minimum height=1.5cm] (J) {$J$};
    \draw[->] (-.2,.5) node[above] {$\theta$}  .. controls  ++(-.15,-.3) and ++(-.15,.3) ..  ++(0,-1);
    \draw[->] (1.4,-.5) node[below] {$\tau$}  .. controls  ++(.15,.3) and ++(.15,-.3) ..  ++(0,1);
    \draw (.5,-1) node {$J\ddot{\theta}=\tau$};
\end{tikzpicture}};
\draw (M0) node {$J_{1}$};
\draw (M1) node {$J_{2}$};


\draw (K1) -| (M1.180);
\draw (D1) -| (M1.180);
\draw (K1.180) -| (M0.0);
\draw (D1.180) -| (M0.0);
\draw[->] (M0.180) ++(0,.5) node[above] {$\tau$} ++(-.1,-.1) .. controls  ++(-.15,-.3) and ++(-.15,.3) ..  ++(0,-.8);
\draw[->] (M0.180) ++(.25,.5) node[above] {$\theta$} ++(-.1,-.1) .. controls  ++(-.15,-.3) and ++(-.15,.3) ..  ++(0,-.8);
\end{tikzpicture}
\end{center}
Find the transfer function from input $\tau$, the applied torque, to output $\theta$, the orientation of the satellite around is major axis. 