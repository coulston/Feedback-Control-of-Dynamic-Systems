The asymptotic loci rule proves this point. Consider the equation for asymptote angles:
\[
\phi_{A} = \frac{2q+1}{n-m}180^{\circ} \quad q = 0,1,\cdots,n-m-1
\]
where $n$ is the number of poles and $m$ is the number of zeros. If the relative degree is 3 or higher, then $n-m-1 \geq 2$, which means that there are at least 3 asymptotes. Since these are evenly spaced, a relative order of 3 produces 3 asymptotes spaced $120^{\circ}$ apart, a relative order of 4 produces 4 asymptotes spaced $90^{\circ}$ apart, and so on. In both cases and for higher relative orders, at least one asymptote must enter into the right half plane (RHP). Since loci approach asymptotes, at least one locus (and therefore at least one closed-loop pole, for $K$ large) must enter the RHP, resulting in an unstable closed-loop system. 