The following shows a unity gain feedback system that is used to control the position of a load using a motor. Let $R_{a}=1$ $\Omega$, $K_{t}=K_{e}=1$, $J=0.5$ kg m$^2$, $k=1$ Nm/rad. (The inductance is negligible, so $L_{a}=0$.)

\resizebox{6in}{!}{
\begin{tikzpicture}[scale=1.75,inner sep=0pt,outer sep=0pt,very thick,sysblock/.style={draw,rectangle,inner sep=2pt,minimum width=1cm,minimum height=1cm,very thick}]
\draw (-1,-1) node (a) {}; 

\draw (-1,0) node (Va) { \input{\mainfolder/DrawingElements/CircuitElements/voltagesource.tex}};
\draw (-1,0) node[left=.35in] (x) {$v_{a}$};
\draw (1,0) node[circle,draw] (Vb) {\rule{0pt}{30pt}};
\draw (0,1) node (R) {\input{\mainfolder/DrawingElements/CircuitElements/resistor.tex}};
\draw (0,1) node[above=.25in] {$R_{a}$};
\draw (0,1) node[below=.15in] {$\begin{matrix} \longrightarrow \\ i_{a}\end{matrix}$};
\draw (2.5,0) node (M1) {\begin{tikzpicture}
    \draw[very thick] (.5,0) node[cylinder,draw,shape aspect=.55,minimum width=1cm,minimum height=1.5cm] (J) {$J$};
    \draw[->] (-.2,.5) node[above] {$\theta$}  .. controls  ++(-.15,-.3) and ++(-.15,.3) ..  ++(0,-1);
    \draw[->] (1.4,-.5) node[below] {$\tau$}  .. controls  ++(.15,.3) and ++(.15,-.3) ..  ++(0,1);
    \draw (.5,-1) node {$J\ddot{\theta}=\tau$};
\end{tikzpicture}};
\draw (2.5,0) node {$J$} ++(0,-.6) node {\small arm};
\draw[->] (M1.135) node[above] (y) {$\theta$} ++(-.1,-.1) .. controls  ++(-.15,-.3) and ++(-.15,.3) ..  ++(0,-.8);
%\draw[->] (M1.135) ++(-.3,0) node[above=2pt] {$\tau$} ++(-.1,-.1) .. controls  ++(-.15,-.3) and ++(-.15,.3) ..  ++(0,-.8);
\draw (4,0) node (K) {\begin{tikzpicture}
\draw (.75,0) node[inner sep=0,outer sep=0] (K1) {\begin{tikzpicture}
\draw (.75,0) node[inner sep=0,outer sep=0] (K1) {\begin{tikzpicture}
\draw (.75,0) node[inner sep=0,outer sep=0] (K1) {\input{\mainfolder/DrawingElements/MechanicalElements/spring.tex}};
\draw (K1)  node[above=6pt] {$k$};
\draw[very thick] (K1.180) -- ++(-.2,0);
\draw[very thick] (K1.0) -- ++(0.2,0);
\draw[<-,thick] (K1.0) ++(.2,0) -- ++(.5,0) node[right] {$f$};
\draw[<-,thick] (K1.180) ++(-.2,0) -- ++(-.5,0) node[left] {$f$};
\draw[|->,thick] (K1.180) ++(-.2,.4) node[above=2pt] {$x_{1}$} -- ++(.5,0);  
\draw[|->,thick] (K1.0) ++(.2,.4) node[above=2pt] {$x_{2}$} -- ++(.5,0);  
\draw<2-> (K1) ++(0,-.6) node {$f=k(x_{1}-x_{2})$};
\end{tikzpicture}
};
\draw (K1)  node[above=6pt] {$k$};
\draw[very thick] (K1.180) -- ++(-.2,0);
\draw[very thick] (K1.0) -- ++(0.2,0);
\draw[<-,thick] (K1.0) ++(.2,0) -- ++(.5,0) node[right] {$f$};
\draw[<-,thick] (K1.180) ++(-.2,0) -- ++(-.5,0) node[left] {$f$};
\draw[|->,thick] (K1.180) ++(-.2,.4) node[above=2pt] {$x_{1}$} -- ++(.5,0);  
\draw[|->,thick] (K1.0) ++(.2,.4) node[above=2pt] {$x_{2}$} -- ++(.5,0);  
\draw<2-> (K1) ++(0,-.6) node {$f=k(x_{1}-x_{2})$};
\end{tikzpicture}
};
\draw (K1)  node[above=6pt] {$k$};
\draw[very thick] (K1.180) -- ++(-.2,0);
\draw[very thick] (K1.0) -- ++(0.2,0);
\draw[<-,thick] (K1.0) ++(.2,0) -- ++(.5,0) node[right] {$f$};
\draw[<-,thick] (K1.180) ++(-.2,0) -- ++(-.5,0) node[left] {$f$};
\draw[|->,thick] (K1.180) ++(-.2,.4) node[above=2pt] {$x_{1}$} -- ++(.5,0);  
\draw[|->,thick] (K1.0) ++(.2,.4) node[above=2pt] {$x_{2}$} -- ++(.5,0);  
\draw<2-> (K1) ++(0,-.6) node {$f=k(x_{1}-x_{2})$};
\end{tikzpicture}
};
%\draw (4,-.5) node (D) {\begin{tikzpicture}
\draw[very thick] (-.2,0) -- (0,0);
\draw (.75,0) node {\begin{tikzpicture}
\draw[very thick] (-.2,0) -- (0,0);
\draw (.75,0) node {\begin{tikzpicture}
\draw[very thick] (-.2,0) -- (0,0);
\draw (.75,0) node {\input{\mainfolder/DrawingElements/MechanicalElements/damper.tex}};
\draw (.75,0) node[above=9pt] {$b$};
\draw[very thick] (1.5,0) -- ++(.2,0);
    \draw[<-,thick] (1.5,0) ++(.2,0) -- ++(.5,0) node[right] {$f$};
    \draw[<-,thick] (-.2,0) -- ++(-.5,0) node[left] {$f$};
    \draw[|->,thick] (-.2,.4) node[above=2pt] {$x_{1}$} -- ++(.5,0);  
    \draw[|->,thick] (1.7,.4) node[above=2pt] {$x_{2}$} -- ++(.5,0);  
    \draw (.6,-.6) node {$x=x_{1}-x_{2}$};
  %  \draw (.6,-1.2) node {$f=b\dot{x}$};
\end{tikzpicture}};
\draw (.75,0) node[above=9pt] {$b$};
\draw[very thick] (1.5,0) -- ++(.2,0);
    \draw[<-,thick] (1.5,0) ++(.2,0) -- ++(.5,0) node[right] {$f$};
    \draw[<-,thick] (-.2,0) -- ++(-.5,0) node[left] {$f$};
    \draw[|->,thick] (-.2,.4) node[above=2pt] {$x_{1}$} -- ++(.5,0);  
    \draw[|->,thick] (1.7,.4) node[above=2pt] {$x_{2}$} -- ++(.5,0);  
    \draw (.6,-.6) node {$x=x_{1}-x_{2}$};
  %  \draw (.6,-1.2) node {$f=b\dot{x}$};
\end{tikzpicture}};
\draw (.75,0) node[above=9pt] {$b$};
\draw[very thick] (1.5,0) -- ++(.2,0);
    \draw[<-,thick] (1.5,0) ++(.2,0) -- ++(.5,0) node[right] {$f$};
    \draw[<-,thick] (-.2,0) -- ++(-.5,0) node[left] {$f$};
    \draw[|->,thick] (-.2,.4) node[above=2pt] {$x_{1}$} -- ++(.5,0);  
    \draw[|->,thick] (1.7,.4) node[above=2pt] {$x_{2}$} -- ++(.5,0);  
    \draw (.6,-.6) node {$x=x_{1}-x_{2}$};
  %  \draw (.6,-1.2) node {$f=b\dot{x}$};
\end{tikzpicture}};
\draw (K) node[above=.25in] {$k$};
%\draw (D) node[above=.25in] {$b$};
\draw (5,0) node[rotate=180] (gnd) {\input{\mainfolder/DrawingElements/MechanicalElements/ground.tex}};

%\draw (M1.0) |- (D.180);
\draw (M1.0) |- (K.180);
%\draw (D.0)  -- ++(0.25,0) |- (gnd.0);
\draw (K.0)  -- ++(0.25,0) |- (gnd.0);
\draw (1,0) -- (M1.180);
\draw (R.0) -| (Vb.90);
\draw (Va.90) |- (R.180);
\draw (Va) -- (a);
\draw (a.0) -| (Vb.-90);


% control
\draw (-2.5,0) node[sysblock]  (C) {$K\frac{s+1}{s}$};
\draw (-3.5,0) node[draw,circle,inner sep=0pt,outer sep=0pt,very thick] (sum) {\rule{12pt}{0pt}};
\draw[thick,dotted,->] (y) -- ++(0,1.1)  -| (sum.90) node[above right=2pt] {$-$};
\draw[thick,dotted,->] (C.0) -- (x.180);
\draw[thick,dotted,->] (sum.0) -- (C.180);
\draw[<-,thick,dotted] (sum.180) node[above left=2pt] {$+$} -- ++(-1,0) node[above] {$\theta_{d}$};
\end{tikzpicture}}

Sketch a root locus of the closed loop poles that can be achieved by varying the gain $K$.

