\mode<presentation>
{
  \usetheme{CambridgeUS}
  \usecolortheme{whale}
  \usecolortheme{lily}

  \setbeamercovered{transparent}
  \usefonttheme[onlymath]{serif}
}

\title[\BodePlotsIVShortName] % (optional, use only with long paper titles)
{\course: \BodePlotsIVName\license}

\subtitle
{Lecture \BodePlotsIVNumber} % (optional)


% Delete this, if you do not want the table of contents to pop up at
% the beginning of each subsection:
%\AtBeginSection[]
%{
%  \begin{frame}<beamer>{Outline}
%    \tableofcontents[currentsection,currentsubsection]
%  \end{frame}
%}


% If you wish to uncover everything in a step-wise fashion, uncomment
% the following command:

%\beamerdefaultoverlayspecification{<+->}


\begin{document}

\begin{frame}
  \titlepage
\end{frame}

\mode<article>{
\maketitle
\tableofcontents
}

%\mode<presentation>{
%\begin{frame}{Outline}
%  \tableofcontents
%  % You might wish to add the option [pausesections]
%\end{frame}}

\section{Systematic Method for Sketching Bode Plots}

We will explain a systematic method for sketching Bode plots, using the system
\begin{frame}
\[
G(s) = \frac{5(s+1)(s-50)}{s^{2}(s^{2}+10s+100)}
\]
\end{frame}
as a working example
\begin{itemize}
\item \begin{frame}
Step 1: Factor out constant terms
\begin{align*}
G(s) &= \frac{5(-50)}{100s^{2}}\frac{(s+1)(\frac{s}{-50}+1)}{\left(\left(\frac{s}{10}\right)^{2}+\frac{s}{10}+1\right)}\\
& = \frac{-2.5}{s^{2}}\frac{(s+1)(\frac{s}{-50}+1)}{\left(\left(\frac{s}{10}\right)^{2}+\frac{s}{10}+1\right)}
\end{align*}
the term $\frac{-2.5}{s^{2}}$ is called the {\em low frequency term}
\end{frame}
\item \begin{frame} Step 2: List break frequencies and important info
\begin{center}
\resizebox{6.5in}{!}{\begin{tabular}{lllll}
\toprule
Break Frequency & Item (P/Z? L/RHP? \#?) & Magnitude Slope & Phase Slope & Range for Phase Slope\\\midrule
1 rad/s & 1 LHP zero & 20 dB/dec & 45$^{\circ}$/dec & 0.1 to 10 rad/s \\
10 rad/s & 2 LHP poles & -40 dB/dec &  -90$^{\circ}$/dec & 1 to 100 rad/s \\
50 rad/s & 1 RHP zero & 20 dB/dec & -45$^{\circ}$/dec & 5 to 500 rad/s \\\bottomrule
\end{tabular}}
\end{center}
\end{frame}
\item \begin{frame} Step 3: Calculate gain and phase of low frequency term ($2.5/s^{2}$). To calculate the magnitude, we pick a frequency less than or equal to all break frequencies. In this case 1 rad/s is convenient, so we plug in $s=j\omega$ with $\omega=1$:
\[
\left|-\frac{2.5}{s^{2}}\right|_{s=j} = \frac{|-2.5|}{|j^{2}|} = \frac{2.5}{1} = 2.5
\]
Thus, at 1 rad/s, the magnitude is $20\log_{10}(2.5) = 7.96$. To calculate low frequency phase, you can always just plug in $j$ 
\[
\angle \left.-\frac{2.5}{s^{2}}\right._{s=j} = \angle \frac{-2.5}{j^{2}} = \angle \frac{-2.5}{-1} = \angle 2.5 = 0^{\circ}
\]
In this case, the low frequency phase is $0^{\circ}$.
\end{frame}
\item Step 4: Draw magnitude plot, starting from lowest frequencies. If the low  frequency term is just a gain, mark that gain at the lowest frequency, otherwise follow the first two steps below

\begin{center}
\begin{minipage}{1.5in}
Plot point at 1 rad/s and 7.96 db
\end{minipage}\hspace{.25in}
\begin{frame}
\mode<presentation>{Step 4: Draw magnitude plot, starting from lowest frequencies.\vspace{.25in}}
\begin{minipage}{3in}
\includegraphics[width=3in]{figures/magprocedure1}
\end{minipage}
\end{frame}


\begin{minipage}{1.5in}
Draw line with slope of -20 dB/dec for each pure integrator term. In this case, the low frequency term is $\frac{2.5}{s^{2}}$, which has two pure integrators, so the initial slope is $-40$ dB/dec
\end{minipage}\hspace{.25in}
\begin{frame}
\begin{minipage}{3in}
\begin{tikzpicture}
\draw (0,0) node {\includegraphics[width=3in]{figures/magprocedure2}};
\draw(-1.78,1.58) node[above right]{$-40$ \textsf{dB/dec}};
\draw[thick] (-1.78,1.58) -- ++(.2,0) -- ++(0,-.15);
\end{tikzpicture}
\end{minipage}
\end{frame}

\begin{minipage}{1.5in}
Since 1 rad/s is a break frequency with one LHP zero, change slope by +20 dB/dec or a net -40+20=-20 dB/dec and extend to next highest break frequency 
\end{minipage}\hspace{.25in}
\begin{frame}
\begin{minipage}{3in}
\begin{tikzpicture}
\draw (0,0) node {\includegraphics[width=3in]{figures/magprocedure3}};
\draw(-1.78,1.58) node[above right]{$-40$ \textsf{dB/dec}};
\draw[thick] (-1.78,1.58) -- ++(.2,0) -- ++(0,-.15);
\draw(-0.15,0.8) node[above right]{$-20$ \textsf{dB/dec}};
\draw[thick] (-0.15,0.8) -- ++(.2,0) -- ++(0,-.1);
\end{tikzpicture}
\end{minipage}
\end{frame}

\begin{minipage}{1.5in}
Continue changing slope after each break frequency as dictated by pole or zero location and number
\end{minipage}\hspace{.25in}
\begin{frame}
\begin{minipage}{3in}
\begin{tikzpicture}
\draw (0,0) node {\includegraphics[width=3in]{figures/magprocedure4}};
\draw(-1.78,1.58) node[above right]{$-40$ \textsf{dB/dec}};
\draw[thick] (-1.78,1.58) -- ++(.2,0) -- ++(0,-.15);
\draw(-0.15,0.8) node[above right]{$-20$ \textsf{dB/dec}};
\draw[thick] (-0.15,0.8) -- ++(.2,0) -- ++(0,-.1);
\draw(1.4,0) node[above right]{$-60$ \textsf{dB/dec}};
\draw[thick] (1.4,0) -- ++(.2,0) -- ++(0,-.2);
\draw[thick] (2.3,-.75) -- ++(.2,0) -- ++(0,-.17);
\draw[thick] (2.3,-.75)  node[above right] {$-40$ \textsf{dB/dec}};
\end{tikzpicture}
\end{minipage}
\end{frame}
\end{center}
When drawing the plot, it can be convenient to calculate the magnitudes at the break frequencies. For example, we know that at 1 rad/s the magnitude is 7.96 dB. Since the linear approximation decreases by -20 dB/dec from this point, the magnitude at 10 rad/s is
\[
\text{Mag at 10 rad/s} = 7.96\mbox{dB}. - 20 \mbox{dB/dec} \times 1 \mbox{decade} = -12.04\mbox{dB}.
\]
The next break frequency is at 50 rad/s. Note that 
\[
\log_{10}(50) = 1.7
\]
and
\[
\log_{10}(10)= 1
\]
Thus 50 rad/s is 1.7-1 = 0.7 of a decade from 10 rad/s. The magnitude at 50 rad/s is then
\[
\text{Mag at 50 rad/s} = -12.04\mbox{dB}. - 60 \mbox{dB/dec} \times 0.7 \mbox{decade} = -54.04\mbox{dB}.
\]

\item Step 5: Indicate regions on phase plot where slope is non-zero. To do this, draw a line above the phase plot for each break frequency listed in the table above. This line will extend from one decade below to one decade above the break frequency, as indicated in the ``Range for Phase Slope'' column from Step 2. On each line, list the slope that is associated with that term. 
\begin{center}
\begin{minipage}{1.5in}
Draw lines centered at 1, 10 and 50 rad/s
\end{minipage}\hspace{.25in}
\begin{minipage}{3in}
\begin{frame}
\mode<presentation>{Step 5: Draw phase plot}
\begin{tikzpicture}
\draw (0,0) node {\includegraphics[width=3in]{figures/phaseproceedure1}};
\draw[thick,*-*] (-2.7,2.1) -- node[pos=.2,above] {45$^{\circ}$/dec} ++(3.67,0);
\draw[thick,*-*] (-.97,2.3) -- node[pos=.15,above] {-90$^{\circ}$/dec} ++(3.67,0);
\draw[thick,*-*] (0.25,2.5) -- node[pos=.2,above] {-45$^{\circ}$/dec} ++(3.67,0);
\end{tikzpicture}
\end{frame}
\end{minipage}
\end{center}
\item Step 6: Draw phase plot, starting from the lowest frequencies. As you go from low to high frequency, you can determine the proper slope by adding up the numbers associated with each line at each frequency.
\begin{center}
\begin{minipage}{1.5in}
From calculation above, low frequency phase is 0$^{\circ}$, so start there, and follow slope indicated by lines
\end{minipage}\hspace{.25in}
\begin{minipage}{3in}
\begin{frame}
\begin{tikzpicture}
\draw (0,0) node {\includegraphics[width=3in]{figures/phaseproceedure2}};
\draw[thick,*-*] (-2.7,2.1) -- node[pos=.2,above] {45$^{\circ}$/dec} ++(3.67,0);
\draw[thick,*-*] (-.97,2.3) -- node[pos=.15,above] {-90$^{\circ}$/dec} ++(3.67,0);
\draw[thick,*-*] (0.25,2.5) -- node[pos=.2,above] {-45$^{\circ}$/dec} ++(3.67,0);
\draw(-1.28,1.32) node[above left]{$45^{\circ}$\textsf{/dec}};
\draw[thick] (-1.28,1.32) -- ++(-.2,0) -- ++(0,-.07);
\end{tikzpicture}
\end{frame}
\end{minipage}
\end{center}
\begin{center}
\begin{minipage}{1.5in}
at 5 rad/s, the phase will be $45 - 45\times (\log_{10}(5) - \log_{10}(1)) = 13.5^{\circ}$
\end{minipage}\hspace{.25in}
\begin{minipage}{3in}
\begin{frame}
\begin{tikzpicture}
\draw (0,0) node {\includegraphics[width=3in]{figures/phaseproceedure3}};
\draw[thick,*-*] (-2.7,2.1) -- node[pos=.2,above] {45$^{\circ}$/dec} ++(3.67,0);
\draw[thick,*-*] (-.97,2.3) -- node[pos=.15,above] {-90$^{\circ}$/dec} ++(3.67,0);
\draw[thick,*-*] (0.25,2.5) -- node[pos=.2,above] {-45$^{\circ}$/dec} ++(3.67,0);
\draw(-1.28,1.32) node[above left]{$45^{\circ}$\textsf{/dec}};
\draw[thick] (-1.28,1.32) -- ++(-.2,0) -- ++(0,-.07);
\draw(-.5,1.32) node[above right]{$-45^{\circ}$\textsf{/dec}};
\draw[thick] (-.5,1.32) -- ++(.2,0) -- ++(0,-.07);
\end{tikzpicture}
\end{frame}
\end{minipage}
\end{center}
\begin{center}
\begin{minipage}{1.5in}
The slope changes as each line ``turns on'' or ``turns off'\end{minipage}\hspace{.25in}
\begin{minipage}{3in}
\begin{frame}
\begin{tikzpicture}
\draw (0,0) node {\includegraphics[width=3in]{figures/phaseproceedure4}};
\draw[thick,*-*] (-2.7,2.1) -- node[pos=.2,above] {45$^{\circ}$/dec} ++(3.67,0);
\draw[thick,*-*] (-.97,2.3) -- node[pos=.15,above] {-90$^{\circ}$/dec} ++(3.67,0);
\draw[thick,*-*] (0.25,2.5) -- node[pos=.2,above] {-45$^{\circ}$/dec} ++(3.67,0);
\draw(-1.28,1.32) node[above left]{$45^{\circ}$\textsf{/dec}};
\draw[thick] (-1.28,1.32) -- ++(-.2,0) -- ++(0,-.07);
\draw(-.5,1.32) node[above right]{$-45^{\circ}$\textsf{/dec}};
\draw[thick] (-.5,1.32) -- ++(.2,0) -- ++(0,-.07);
\draw(.5,1) node[above right]{$-90^{\circ}$\textsf{/dec}};
\draw[thick] (.5,1) -- ++(.2,0) -- ++(0,-.15);
\draw(1.6,.1) node[above right]{$-135^{\circ}$\textsf{/dec}};
\draw[thick] (1.6,.1) -- ++(.2,0) -- ++(0,-.2);
\draw(3,-.86) node[above right]{$-45^{\circ}$\textsf{/dec}};
\draw[thick] (3,-.86) -- ++(.2,0) -- ++(0,-.07);
\end{tikzpicture}
\end{frame}
\end{minipage}
\end{center}

\item Step 7: Verify your phase plot by calculating the total phase \textit{change} for each of the poles and zeros in your Table from Step 2.

\begin{center}
	{\begin{tabular}{lllll}
			\toprule
			Break Frequency & Item (P/Z? L/RHP? \#?) & Phase Slope & \# of Decades & Phase Change for Item \\\midrule
			1 rad/s & 1 LHP zero & 45$^{\circ}$/dec & 2 decades & 90$^{\circ}$\\
			10 rad/s & 2 LHP poles &  -90$^{\circ}$/dec & 2 decades & -180$^{\circ}$\\
			50 rad/s & 1 RHP zero & -45$^{\circ}$/dec & 2 decades & -90$^{\circ}$\\\midrule
			 & & & Total Phase Change & -180$^{\circ}$\\\bottomrule
	\end{tabular}}
\end{center}

In this example, the total phase change should be therefore be -180$^{\circ}$, which is consistent with the phase starting at 0$^{\circ}$ and ending at -180$^{\circ}$ in the final plot of Step 6. 

\end{itemize}


\section{Lecture Highlights}
The primary takeaways from this article include
\begin{enumerate}
\setlength{\itemsep}{5pt}
\setlength{\parskip}{0pt}
\setlength{\parsep}{0pt}
\item This article gives a systematic method for sketching Bode plots that builds on all of the rules derived in the previous three articles. 
\item The systematic method requires identifying break frequencies (sometimes called corner frequencies or critical frequencies) and determining how the magnitude and phase bode plots change slope at (for magnitude) or around (for phase) those frequencies.
\end{enumerate}


\section{Quiz Yourself}

\subsection{Questions}

\begin{enumerate}
\setlength{\itemsep}{5pt}
\setlength{\parskip}{0pt}
\setlength{\parsep}{0pt}
\item Sketch the Bode plot for the following systems. Label your sketch with the magnitude and phase at each break point. Verify your results using MATLAB. In order to select the frequency range of the Bode plot, you can use the second argument: \texttt{bode(sys,\{wmin, wmax\})}, where you replace \texttt{wmin} with the minimum desired frequency and \texttt{wmax} with the maximum frequency. You will need to include the curly brackets.
\begin{enumerate}
\item
\[
G(s) = \frac{(s+10)}{s(s+1)(s+2)(s+20)}
\]
\item 
\[
G(s) = \frac{10(s-1)}{s(s^{2}+20s+100)}
\]
\item 
\[
G(s) = \frac{900(s+10)}{(s+30)(s^{2}-s+1)}
\]
\end{enumerate}
\end{enumerate}
\subsection{Solutions}
\begin{enumerate}
\setlength{\itemsep}{5pt}
\setlength{\parskip}{0pt}
\setlength{\parsep}{0pt}
\item[1a] \rule{12pt}{0pt}
\begin{center}
\includegraphics[width=6in]{quizfigures/1asolnc}\\
\includegraphics[width=6in]{quizfigures/1bsoln}\\
\includegraphics[width=6in]{quizfigures/1csoln}\\
\includegraphics[width=5in]{quizfigures/1dsoln}
\end{center}
\newpage
\item[1b] \rule{12pt}{0pt}
\begin{center}
\includegraphics[width=5in]{quizfigures/2asolnc}\\
\includegraphics[width=6in]{quizfigures/2bsoln}\\
\includegraphics[width=5in]{quizfigures/2csoln}
\end{center}
\item[1c] \rule{12pt}{0pt}
\begin{center}
\includegraphics[width=5in]{quizfigures/3asolnc}\\
\includegraphics[width=6in]{quizfigures/3bsoln}\\
\includegraphics[width=5in]{quizfigures/3csoln}
\end{center}
\end{enumerate}


\end{document}


