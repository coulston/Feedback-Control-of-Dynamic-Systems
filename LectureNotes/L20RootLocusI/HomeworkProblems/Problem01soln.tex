\begin{enumerate}[(a)]
\item \rule{0pt}{0pt}\\
\includegraphics[width=6in]{\mainfolder/LectureNotes/\lecturefolder/HomeworkProblems/Problem01soln1}\\ 

\includegraphics[width=6in]{\mainfolder/LectureNotes/\lecturefolder/HomeworkProblems/Problem01soln2}
\item Adding a zero will mean the pole excess is 2, and thus the asymptotes will be at $\pm 90^{\circ}$, Since the term $(s+a)$ would provide a zero at $s=-a$, the equation for the asymptote center is
\[
\sigma = \frac{ -5 -1 -5 - (-a)}{3-1} = \frac{-7+a}{2}
\]
To have $\sigma=-2$ we would choose $a=3$. 
\end{enumerate}