The following shows a feedback control system where the height of the tank is to be controlled. The fluid has density $\rho =1$ and take $g=9.81$.  This is a proportional feedback control system, with desired reference $r$ and proportional gain 20. However, the measurement is transmitted over a network that causes a delay of $T_{d}$ seconds. Estimate the maximum delay that can be tolerated before the closed loop system becomes unstable. You will need to find the transfer function $\frac{Y(s)}{U(s)}$ (or equivalently $\frac{H(s)}{Q_{in}(s)}$) and sketch the Bode plot to answer this question. A log scale is provided for your convenience, as is the impedance network for the tank.

\begin{center}
\begin{minipage}{4.5in}
\resizebox{4.5in}{!}{
\begin{tikzpicture}
\draw (.75,0) node[above] (tank) {\input{\mainfolder/DrawingElements/FluidElements/tank.tex}};
\draw[decorate,decoration={coil,aspect=0,segment length=5.85pt}] (-.45,2.25) -- ++(2.38,0);
\draw (-1.15,1) node (pipe1) {\input{\mainfolder/DrawingElements/FluidElements/pipe.tex}};
\draw (2.65,1) node (pipe2) {\input{\mainfolder/DrawingElements/FluidElements/valve.tex}};

\draw (1.7,3) node[inner sep=0,outer sep=0] (meas) {\begin{tikzpicture} \draw[very thick] (0,0) -- ++(.1,0) -- ++(0,-.1) -- ++(.2,0) -- ++(0,.1) -- ++(.1,0) -- ++(0,.4) -- ++(-.4,0) -- cycle;\end{tikzpicture}};
\draw (-6,1) node[draw,circle,very thick] (sum) {\rule{9pt}{0pt}};
\draw (-4.25,1) node[draw,rectangle,very thick,minimum width=.5in,minimum height=.5in,outer sep=0pt] (C) {$20$};
\draw (-4.25,3.6) node[draw,rectangle,very thick,outer sep=0pt] (D) {$T_{d}$ sec \textsf{Delay}};

\draw[->,thick,dotted] (meas.90) |- node[above left] {measurement of $y=h$} (D);
\draw[->,thick,dotted] (D) -| (sum.90) node[above right] {$-$};
\draw[<-,thick,dotted] (sum.180) node[above left] {$+$} -- ++(-1,0) node[above] {$r$};
\draw[->,thick,dotted] (sum.0) -- (C.180);
\draw[->,thick,dotted] (C.0) -- ++(.45,0);



\draw[->] (.2,.75) -- node[pos=.5,left] {$h$} ++(0,1.4);
\draw (tank.-90) node{Tank Area: $2$ m$^{2}$};
\draw (pipe2.90) node[above] {$\frac{9.81}{4}$};
\draw (.75,.8) node[above] {$p_{1}$};
\draw[<-] (pipe1.180) ++(.9,0) --  ++(-.5,0) node[left] {$u=q_{in}$};
\draw[->] (pipe2.0) ++(-.5,0) --  ++(.5,0) node[right] {$q_{out}$};
\draw (.75,2.5) node[above] {$p_{a}$};
\draw (pipe2.0) ++(1.5,0) node {$p_{a}$};
\draw (pipe2.0) ++(1.5,2) node {$p_{1} = \rho g h$};
\end{tikzpicture}}
\end{minipage}
\begin{minipage}{2in}
\resizebox{2in}{!}{
\begin{tikzpicture}
\draw (-2,-1.5) node[scale=.85,inner sep=0pt,outer sep=0pt] (a) {\input{\mainfolder/DrawingElements/CircuitElements/currentsource.tex}};
\draw (-.5,0) node[circle,fill=black,inner sep=0,minimum width=4pt] {} node[above=2pt] {$P_{1}$};
\draw (a) node[left=9pt] {$Q_{in}(s)$};
\draw (-.5,-1.5) node[rectangle,draw,minimum width=.1in,minimum height=.5in] (tank1) {};
\draw (tank1) node[right=9pt] {$\frac{\rho g}{sA}$};
\draw (1,0) node[rectangle,draw,minimum width=.5in,minimum height=.1in] (valve1) {};
\draw (valve1) node[above=9pt] {$R$};
\draw (2.5,-3) node[inner sep=0] (pa) {};
\draw (-.5,-3) node[circle,fill=black,minimum width=4pt,inner sep=0] {};
\draw (-.5,-3) node[below=4pt,circle, inner sep=1pt] {$P_{a}$};
\draw (2.5,-3) node[inner sep=0,outer sep=0] (pa2) {};
\draw (-.5,-3) node[inner sep=0,outer sep=0] (pa3) {};
\draw[very thick] (a.90) |- (valve1.180);
\draw[very thick] (tank1.90) |- (valve1.180);
\draw[very thick] (valve1.0) -| (pa) -- (pa3);
\draw[very thick] (tank1.-90) -- (pa3) -| (a.-90);
%\draw[->] (3,-1) -- node[pos=.5,right] {$Q_{out}(s)$} ++(0,-1);
\end{tikzpicture}}
\end{minipage}
\includegraphics[width=5in]{\mainfolder/LectureNotes/\lecturefolder/HomeworkProblems/Problem06/blankbode}
\end{center}
