\usepackage{enumitem}

\mode<presentation>
{
  \usetheme{CambridgeUS}
  \usecolortheme{whale}
  \usecolortheme{lily}

  \setbeamercovered{transparent}
  \usefonttheme[onlymath]{serif}
}

\title[\ApplicationExampleIShortName] % (optional, use only with long paper titles)
{\course: \ApplicationExampleIName\license}

\subtitle
{Lecture \ApplicationExampleINumber} % (optional)



\begin{document}

\begin{frame}
  \titlepage
\end{frame}

\mode<article>{
\maketitle
\tableofcontents
}

%\mode<presentation>{
%\begin{frame}{Outline}
%  \tableofcontents
%  % You might wish to add the option [pausesections]
%\end{frame}}
%\section{\color{red}{Section notes to self - delete before posting}}
%\begin{itemize}
%	\item \color{red}{I'm thinking about making this one about solar panel orientation control to go along with the Simulink video that I didn't play in Lecture 1: \url{https://www.youtube.com/watch?v=bE179wgm164}. One idea is to play a portion of the video, then discuss class concepts that come up. Did Elenya and Hisham play this in their sections on the 1st day?}
%	\item \color{red}{Check AQ1-2 results and use this lecture to emphasize concepts that might have had low scores}
%	\item \color{red}{Consider asking: how would you model? (motor lecture - electrical and rotational systems)}
%	\item \color{red}{Consider asking: what feature(s) of the model suggest potential control problems, if any? (consider speed, resonance, stability, etc.)}
%	\item \color{red}{Based on answer to above, what type(s) of controller(s) might be useful to apply?}
%	\item \color{red}{Consider asking: who cares? (why do we want this)}
%	\item \color{red}{Consider asking: is this in the news lately? how would it impact you or your community?}
%	\item \color{red}{Consider asking: who might have written this problem statement? How might things (e.g., specifications, solutions, ...) be different if someone else had written it? (e.g., someone who lives in a community where solar panel materials are mined, someone who owns a natural gas field, ...)}
%	\item \color{red}{Consider asking: what Matlab tips did you learn from the video that will help your team with the project?}
%	\item \color{red}{update prerequisite material list}
%\end{itemize}

%\begin{frame}{\textcolor{red}{Frame Content for Presentation File}}
%\begin{center}
%\mode<article>{\includegraphics[width=4.5in]{figures/emptyfig.png}
}
%\mode<presentation>{\resizebox{3in}{!}{\includegraphics[width=4.5in]{figures/emptyfig.png}
}}
%\end{center}
%\end{frame}


\section{The Big Picture: Evaluating an Application for Control Design}

Let's take a step back to think about control design a bit more holistically for a specific example. As control engineers, we're interested in being able to model a system mathematically (so that we can analyze it to design a controller) and to incorporate specifications into our designs, both of which we have done for a number of examples this semester. However, we might also want to think in a more ``big picture'' manner, especially to question the assumptions we are given. We might ask such questions as:

\begin{itemize}
\setlength{\itemsep}{0pt}
\setlength{\parskip}{0pt}
\setlength{\parsep}{0pt}
	\item For whom are we creating a design? There might be an obvious client, but are there other stakeholders that might not yet be considered? Are there people, communities, environments, or flora/fauna that might be impacted if the project succeeds...or if it fails? Are there things we can do as control engineers to minimize any negative impacts?
	\item How is the ``problem'' we are trying to solve currently defined? Are there other ways we could think about it, and how would that impact our designs?
	\item What do we need to model in order to create a design? How do we know when a model is ``good enough''?
	\item Given our model, what kinds of issues do we anticipate?
	\item How will we measure success?
	\item and more.
\end{itemize}

This ``article'' is a bit unusual as it is more of a scaffold for class discussion than an article itself.

\section{Motion Modeling and Control for a Solar Array using Simulink}

Begin by opening the video by the Mathworks at \url{https://www.youtube.com/watch?v=bE179wgm164}. This 11-min video illustrates the process of modeling, designing, and evaluating a controller for a solar panel consisting of a motor with rotating element that tracks the sun across the sky. It is produced as a teaching tool for Simulink but can illustrate many of the concepts we are working on in this class.

\subsection{Motivation and Overview}

Watch the first 1:30 minutes of the video and answer these questions:
\begin{enumerate}
\setlength{\itemsep}{60pt}
\setlength{\parskip}{0pt}
\setlength{\parsep}{0pt}
	\item What reason is given for needing an active control system? In other words, what is the problem that we are trying to solve?
	\item What are the three main steps listed for the process that will be illustrated?
	\item In what domain (time or frequency) is the physical system initially modeled? Can you see any similarities to the DC motor or rotating mechanical systems models we have built in this class?
\end{enumerate}
\vspace{80pt}

\subsection{Panel Model Creation}

Since this is a Simulink tutorial video, it should come as no surprise that Simulink is used to build the model. Watch the next 4:45 minutes of the video (until time 6:15) and answer these questions:
\begin{enumerate}[resume]
\setlength{\itemsep}{60pt}
\setlength{\parskip}{0pt}
\setlength{\parsep}{0pt}
	\item What do you notice about building the model from the time-domain equations (from the ``Motivation and Overview'' part of this application example) vs. by first deriving your transfer function and implementing that in Simulink directly like we have previously done?
	\item Given that the Simulink model consists of integrators, gains, and a summing junction (square version), could you use block diagram techniques to derive the transfer function for the panel equation of motion from torque to angular position theta?
	\item Given your understanding of stability and what you observe in the video, which of the following two systems would you expect to be BIBO stable and which not? Why?
	\begin{itemize}
		\item System with torque as input and angular position (theta) as output
		\item System with torque as input and angular velocity (theta\_dot) as output
	\end{itemize}
	\item Do you notice any new Simulink elements or techniques that might be helpful (beyond what your team has been using for the project thus far)? If so, make a note of them so you don't forget.
\end{enumerate}
\vspace{70pt}

\subsection{Control Design}

The creation of the motor model is summarized very quickly, and then the video gets into control design concepts. 
Watch the next 4:10 minutes of the video (until time 10:25) and answer these questions:
\begin{enumerate}[resume]
\setlength{\itemsep}{60pt}
\setlength{\parskip}{0pt}
\setlength{\parsep}{0pt}
	\item Before designing the controller, what test(s) do they show to check if their plant system (motor + panel) is configured correctly?
	\item How is the reference (``sun position'') initially modeled (i.e., what kind of signal is used)?
	\item What kind of controller is used? What is the reason given for eliminating one of the P, I, or D terms?
	\item The video shows some actual sun data used as a reference. If you didn't have that data, what kind of reference (``sun position'') would better represent the actual movement of the sun, compared to the initial kind of signal selected? What system \textbf{Type} would you expect to need to track this kind of signal with zero steady-state error? \vspace{30pt}
	\item Is the controller's performance deemed suitable? What reasons are given for why or why not?
\end{enumerate}
\vspace{60pt}

\subsection{Summary}

In the last minute of the video, some additional tools and information are mentioned. For example, we sometimes use/teach Simscape in this class, which allows us to obtain models for physical systems without first needing to derive the equations. Feel free to check these out if they seem interesting!

Now that we've finished the video, let's go back to the initial ``big picture'' questions and discuss:
\begin{enumerate}[resume]
\setlength{\itemsep}{80pt}
\setlength{\parskip}{0pt}
\setlength{\parsep}{0pt}
	\item For whom was the video creating a design? Are there other stakeholders that might not yet be considered? Are there people, communities, environments, or flora/fauna that might be impacted if the project succeeds...or if it fails? Are there things we can do as control engineers to minimize any negative impacts?
	\item Is the application example timely and relevant in our (or your) community? Can you envision this kind of control application benefiting you, whether singular or plural? %Do you see similar things in the news?
	\item Are there other ways we could think about the problem of tracking the sun to increase energy efficiency, and how would that impact our designs?
\end{enumerate}
\vspace{80pt}

%\section{Lecture Highlights}
%The primary takeaways from this article include
%\begin{enumerate}
%\setlength{\itemsep}{5pt}
%\setlength{\parskip}{0pt}
%\setlength{\parsep}{0pt}
%\item \textcolor{red}{Make list here...}
%\end{enumerate}
%
\section{Quiz Yourself}

Since this article is written as a guided in-class discussion for a solar panel application example, there are no quiz yourself problems. 
%\subsection{Questions}
%\begin{enumerate}
%\setlength{\itemsep}{5pt}
%\setlength{\parskip}{0pt}
%\setlength{\parsep}{0pt}
%\item \input{quizfigures/quizyourself1.tex} 
%\end{enumerate}
%
%\subsection{Solutions}
%
%\begin{enumerate}
%\setlength{\itemsep}{5pt}
%\setlength{\parskip}{0pt}
%\setlength{\parsep}{0pt}
%\item \rule{0pt}{12pt}\\  % use this \rule command to correctly align solutions given as images
%\begin{center}
%\includegraphics[width=4.5in]{quizfigures/emptyfig.png}
%\end{center}\newpage
%\end{enumerate}



\end{document}


