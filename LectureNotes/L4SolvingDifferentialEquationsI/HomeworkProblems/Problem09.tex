You are studying the behavior of a car bumper during low velocity crashes. The car is modeled as a mass, and the bumper has a spring and damper (assuming no plastic deformation at low velocity). 
\begin{center}
\begin{tikzpicture}[inner sep=0pt,outer sep=0pt,very thick]
\draw (0,0) node (M){\input{\mainfolder/DrawingElements/MechanicalElements/mass2.tex}};
\draw (2,.5) node (K) {\begin{tikzpicture}
\draw (.75,0) node[inner sep=0,outer sep=0] (K1) {\begin{tikzpicture}
\draw (.75,0) node[inner sep=0,outer sep=0] (K1) {\begin{tikzpicture}
\draw (.75,0) node[inner sep=0,outer sep=0] (K1) {\input{\mainfolder/DrawingElements/MechanicalElements/spring.tex}};
\draw (K1)  node[above=6pt] {$k$};
\draw[very thick] (K1.180) -- ++(-.2,0);
\draw[very thick] (K1.0) -- ++(0.2,0);
\draw[<-,thick] (K1.0) ++(.2,0) -- ++(.5,0) node[right] {$f$};
\draw[<-,thick] (K1.180) ++(-.2,0) -- ++(-.5,0) node[left] {$f$};
\draw[|->,thick] (K1.180) ++(-.2,.4) node[above=2pt] {$x_{1}$} -- ++(.5,0);  
\draw[|->,thick] (K1.0) ++(.2,.4) node[above=2pt] {$x_{2}$} -- ++(.5,0);  
\draw<2-> (K1) ++(0,-.6) node {$f=k(x_{1}-x_{2})$};
\end{tikzpicture}
};
\draw (K1)  node[above=6pt] {$k$};
\draw[very thick] (K1.180) -- ++(-.2,0);
\draw[very thick] (K1.0) -- ++(0.2,0);
\draw[<-,thick] (K1.0) ++(.2,0) -- ++(.5,0) node[right] {$f$};
\draw[<-,thick] (K1.180) ++(-.2,0) -- ++(-.5,0) node[left] {$f$};
\draw[|->,thick] (K1.180) ++(-.2,.4) node[above=2pt] {$x_{1}$} -- ++(.5,0);  
\draw[|->,thick] (K1.0) ++(.2,.4) node[above=2pt] {$x_{2}$} -- ++(.5,0);  
\draw<2-> (K1) ++(0,-.6) node {$f=k(x_{1}-x_{2})$};
\end{tikzpicture}
};
\draw (K1)  node[above=6pt] {$k$};
\draw[very thick] (K1.180) -- ++(-.2,0);
\draw[very thick] (K1.0) -- ++(0.2,0);
\draw[<-,thick] (K1.0) ++(.2,0) -- ++(.5,0) node[right] {$f$};
\draw[<-,thick] (K1.180) ++(-.2,0) -- ++(-.5,0) node[left] {$f$};
\draw[|->,thick] (K1.180) ++(-.2,.4) node[above=2pt] {$x_{1}$} -- ++(.5,0);  
\draw[|->,thick] (K1.0) ++(.2,.4) node[above=2pt] {$x_{2}$} -- ++(.5,0);  
\draw<2-> (K1) ++(0,-.6) node {$f=k(x_{1}-x_{2})$};
\end{tikzpicture}
};
\draw (2,-.5) node (D) {\begin{tikzpicture}
\draw[very thick] (-.2,0) -- (0,0);
\draw (.75,0) node {\begin{tikzpicture}
\draw[very thick] (-.2,0) -- (0,0);
\draw (.75,0) node {\begin{tikzpicture}
\draw[very thick] (-.2,0) -- (0,0);
\draw (.75,0) node {\input{\mainfolder/DrawingElements/MechanicalElements/damper.tex}};
\draw (.75,0) node[above=9pt] {$b$};
\draw[very thick] (1.5,0) -- ++(.2,0);
    \draw[<-,thick] (1.5,0) ++(.2,0) -- ++(.5,0) node[right] {$f$};
    \draw[<-,thick] (-.2,0) -- ++(-.5,0) node[left] {$f$};
    \draw[|->,thick] (-.2,.4) node[above=2pt] {$x_{1}$} -- ++(.5,0);  
    \draw[|->,thick] (1.7,.4) node[above=2pt] {$x_{2}$} -- ++(.5,0);  
    \draw (.6,-.6) node {$x=x_{1}-x_{2}$};
  %  \draw (.6,-1.2) node {$f=b\dot{x}$};
\end{tikzpicture}};
\draw (.75,0) node[above=9pt] {$b$};
\draw[very thick] (1.5,0) -- ++(.2,0);
    \draw[<-,thick] (1.5,0) ++(.2,0) -- ++(.5,0) node[right] {$f$};
    \draw[<-,thick] (-.2,0) -- ++(-.5,0) node[left] {$f$};
    \draw[|->,thick] (-.2,.4) node[above=2pt] {$x_{1}$} -- ++(.5,0);  
    \draw[|->,thick] (1.7,.4) node[above=2pt] {$x_{2}$} -- ++(.5,0);  
    \draw (.6,-.6) node {$x=x_{1}-x_{2}$};
  %  \draw (.6,-1.2) node {$f=b\dot{x}$};
\end{tikzpicture}};
\draw (.75,0) node[above=9pt] {$b$};
\draw[very thick] (1.5,0) -- ++(.2,0);
    \draw[<-,thick] (1.5,0) ++(.2,0) -- ++(.5,0) node[right] {$f$};
    \draw[<-,thick] (-.2,0) -- ++(-.5,0) node[left] {$f$};
    \draw[|->,thick] (-.2,.4) node[above=2pt] {$x_{1}$} -- ++(.5,0);  
    \draw[|->,thick] (1.7,.4) node[above=2pt] {$x_{2}$} -- ++(.5,0);  
    \draw (.6,-.6) node {$x=x_{1}-x_{2}$};
  %  \draw (.6,-1.2) node {$f=b\dot{x}$};
\end{tikzpicture}};
\draw (3.1,1) -- (3.1,-1);
\draw (M.0) -- ++(.2,0)  |- (K.180);
\draw (M.0) -- ++(.2,0) |- (D.180);
\draw (3.1,0) -- ++(-.2,0)  |- (K.0);
\draw (3.1,0) -- ++(-.2,0) |- (D.0);
\draw[pattern=bricks]  (3.2,0) -- ++(0,1) -- ++(.5,0) -- ++(0,-2) -- ++(-.5,0) -- cycle;
\draw[|->,thick] (M.90) ++(0,.2) node[above=6pt] {$x$} -- ++(.5,0);  
\draw (M) node {$m$};
\draw (K.90) node[above=4pt] {$k$};
\draw (D.-90) node[below=4pt] {$b$};
 
\end{tikzpicture}
\end{center}
When the bumper comes in contact with the barrier, the system is modeled by the differential equation
\[
m \ddot{x} + b\dot{x} + k x = 0
\]
Where $x$ is the position of the car relative to when contact is made (so that $x(0)=0$), $m$ is the car mass, $b$ is the bumper damping coefficient, and $k$ is the bumper spring constant. Suppose $m=1000$ kg, $b=2000$ N s m$^{-1}$, and $k=10,000$ N m$^{-1}$, and the initial velocity of the car is $\dot{x}(0) = 2$ m s$^{-1}$. 

\begin{enumerate}[(a)]
\item Find $x(t)$, $t\geq0$.
\item The solution is only valid for as long as the bumper remains in contact with the wall. For how long is the solution valid? 
\end{enumerate}