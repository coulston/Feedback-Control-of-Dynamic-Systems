The input signal has Laplace transform $U(s) = \frac{1}{s + 3}$.  Therefore,
\[
Y(s) = \frac{s+2}{(s-1)(s+3)^2}
\]
and we can use partial fraction expansion as usual.  
\[
\begin{aligned}
Y(s) &= \frac{A}{s-1} + \frac{B}{s+3} + \frac{C}{(s+3)^2}\\
s+2 &= A(s+3)^2 + B(s-1)(s+3) + C(s-1)\\
s&=1: 3 = 16A \Rightarrow A = \frac{3}{16}\\
s&=-3: -1 = -4C \Rightarrow C = \frac{1}{4}\\	
\end{aligned}
\]
Instead of picking a different number, we can switch techniques and equate coefficients to solve for $B$, using the coefficients on $s^2$:
\[
s^2: 0 = A + B
\]
which tells us that $B = -\frac{3}{16}$.  Therefore, we obtain
\[
y(t) = \frac{3}{16}e^t - \frac{3}{16}e^{-3t} + \frac{1}{4}te^{-3t}, t \geq 0
\]

Note that, even with an input that decays toward zero as $t \rightarrow \infty$, the output increases to infinity due to the $e^t$ term.  This will become important during the semester.

